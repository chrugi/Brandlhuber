\documentclass[a4paper,ngerman,11pt]{scrartcl}

\usepackage[utf8]{inputenc} \usepackage[T1]{fontenc} \usepackage{ngerman}

%\addtokomafont{dispositon}{\ttfamily}

\title{Episode I}
\subtitle{Genesis}
\author{Sahra Roman \& Christian Sangvik}
\date{\today}

\renewcommand{\familydefault}{\ttdefault}
\renewcommand{\ttdefault}{pcr}

\newcommand{\linia}{\rule{.8\textwidth}{.1pt}}

\makeatletter
\renewcommand{\maketitle}{ \texttt{
\begin{center}
  \uppercase{\tiny{
    ETH Zürich
    \enspace{}|\enspace{}D-ARCH
    \enspace{}|\enspace{}Professur Arno Brandlhuber
    \enspace{}|\enspace{}Herbstsemester 2017}}\\
    \linia{}\\
    \vspace*{7cm} \uppercase{\Huge{De Architectura}}\\
    \vspace*{2cm} \uppercase{\normalsize{von}}\\
    \vspace*{1cm} \uppercase{\LARGE{\MakeUppercase{\@author}}}\\
  \clearpage
    \vspace*{3cm} \textbf{\huge{\MakeUppercase{\@title}}}\\
    \vspace*{1cm}
    \Large{\textit{---\enspace{}\MakeUppercase{\@subtitle}\enspace{}---}}\\
    \vspace*{2cm}
\end{center}
}}
\makeatother

\newcommand{\desc}[1]{
  \vspace*{12pt}
  \noindent{}{#1}\\
  \vspace*{12pt} }

\newcommand{\dialog}[2]{
  \begin{center}
  \uppercase{#1}\\
  \vspace*{6pt}
  \begin{minipage}{.8\textwidth}
  {#2}\\
  \end{minipage}\hfill
  \end{center}
  \vspace*{12pt} }

\newcommand{\scene}[4]{
  \vspace*{12pt}
  \noindent{}\textbf{\uppercase{#1}.\quad{}\uppercase{#2}\quad{}-\quad{}\uppercase{#3}}\\
  \desc{#4} }

\newcommand{\fadein}{
  \noindent\uppercase{Fade in:}\\ }

\newcommand{\fadeout}{
  \noindent\uppercase{Fade out:}\\ }

\newcommand{\trans}[1]{
  \hfill\uppercase{{#1}:}\\ }

\newcommand{\subh}[1]{
  \vspace*{12pt}
  \noindent\uppercase{{#1}\\} 
  \vspace*{12pt} }

\begin{document}
\begin{flushleft}

\maketitle

\fadein

\scene{int}{Jans Zimmer}{Morgen}{JAN wacht auf und merkt, dass er verschlafen
  hat.  Er hat die Nacht an seinem Schreibtisch verbracht, wo er beim späten
  Arbeiten eingeschlafen ist. Im Stress packt er seine Sachen zusammen und eilt
  ohne Frühstück aus dem Haus.}

\scene{int}{Bus}{Morgen}{JAN sitzt müde im Bus und schaut aus dem Fenster. Er
  sieht die Stadt Zürich von oben im morgendlichen Nebel.}

\scene{ext}{ETH Hönggerberg}{Morgen}{JAN trifft auf TIM. TIM wartet bereits vor
dem Haupteingang des HIL Gebäudes. Er raucht eine Zigarette und hält einen Becher Kaffee in
der Hand. Als JAN TIM erreicht, betreten sie schnell das Gebäude.}
\dialog{Tim}{Na schau mal an, wer da kommt! Hast du's auch noch geschafft?}{grinsend}
\dialog{Jan}{Du hättest mir lieber auch einen Kaffee mitgebracht.}{murmelnd}
\desc{TIM gibt JAN grinsend den zweiten Becher Kaffee, den er versteckt gehalten hat. Sie betreten das Gebäude.}

\scene{int}{Korridor - ETH}{Morgen}{Onetake - JAN und TIM gehen zusammen durch die
  Korridore in Richtung Koje.}
\dialog{Jan}{Ich bin gestern am Schreibtisch eingeschlafen. Ich komme einfach nicht weiter mit meinem Projekt.\\Vielen Dank, dass du mir dabei hilfst.}
\dialog{Tim}{Kein Problem. Ich habe ja sonst nichts zu tun.}
\dialog{Jan}{Shut up, Bitch!}
\desc{Nach einer kurzen Pause}
\dialog{Jan}{Wie hast du }<++>

--- dialog --- Beziehung TIM-JAN soll klar werden. JANs Problem mit seinem
Projekt und Computer wird angesprochen. TIM soll ihm helfen (ist zuversichtlich)
Projektaufgabe soll auch klar werden.

\subh{ALESSIA kommt dazu}

--- dialog --- Veränderte Haltung von JAN zeigt sein Interesse an ALESSIA. TIM
findet es lustig.

\scene{int}{GUDZILLAs Büro}{Vormittag}{Dr. BROWN sitzt ungeduldig in GUDZILLAs
  Büro und wartet. GUDZILLA tritt ein.}

--- dialog --- GUDZILLA muss BROWN schlechte Nachrichten überbbringen. BROWN hat
etwas anderes erwartet.

\scene{int}{Koje - ETH}{Vormittag}{JAN sitzt frustriert vor seinem Computer. Das
  Programm Dreamcatcher stürzt immer wieder ab. Er bittet TIM um Hilfe, aber der
  kann ihm auch nicht helfen. Sie wenden sich an ALESSIA, da beide wissen wie
  gut sie mit Computern umgehen kann.}

\dialog{Jan}{Mein Computer stürzt ständig ab, noch bevor ich die Parameter speichern kann.}<++>

--- dialog --- Schlussfolgerung ALESSIA, die Vorschlägt, dass er doch zum
Helpdesk am Nachmittag gehen soll.

\scene{int}{Helpdesk - ETH}{Nachmittag}{Dr. BROWN ist schlecht gelaunt. Er sitzt
  am Helpdesk und erklärt gerade einem STUDENTEN, dass er ihm nicht helfen
  kann. JAN wartet ab bis er an der Reihe ist und erklärt dann sein Problem.}

--- dialog --- JAN erklärt sein Problem, BROWN ist zuerst genervt, entwickelt
aber immer mehr Interesse für JANs Problem. Er muss den Laptop bis am Abend bei
sich behalten um das Problem zu beheben. BROWN macht ihm einen Gefallen und
arbeitet länger heute. Er wird JAN per Telefon benachrichtigen, wenn der
Computer bereit ist. JAN wird noch mehr frustriert, gibt sich dann aber
geschlagen.

\scene{int}{Alumni Lounge - ETH}{Abend}{JAN sitzt frustriert und müde am
  Tisch. Neben sich einen leeren Kaffeebecher. Er schaut immer wieder auf sein
  Handy und wartet gestresst auf den Anruf von BROWN. Gedankenverloren kritzelt
  er in ein Notizbuch. Endlich klingelt das Telefon. Der Computer ist
  abholbereit. JAN packt hastig seine Sachen zusammen und geht.}

\scene{int}{Helpdesk - ETH}{Abend}{BROWN übergibt JAN den Laptop. JAN ist
  sichtlich erleichtert.}

--- dialog --- BROWN hat die neue Version installiert. JAN ist glücklich.

\trans{dissolve to}

\scene{int}{JANs Zimmer}{Nacht}{JAN startet seinen Computer und dann auch gleich
  das Programm. Man sieht ihm an, dass er eigentlich überhaupt keine Motivation
  mehr hat und lieber gleich schlafen gehen würde. Sobald das Programm gestartet
  ist, fängt MIRA an zu sprechen.}

\dialog{Mira}{Hallo! Ich bin Mira, Ihre neue Assistentin für den architektonischen Entwurf. Wie kann ich Ihnen helfen?}

--- dialog --- MIRA stellt sich vor, und fragt wie sie helfen kann.

\desc{JAN zögert zunächst. Er ist überfordert mit der neuen, viel
  selbstständigeren Version von Dreamcatcher. Bald aber erkennt er, dass es doch
  noch eine Chance für ein gutes Projekt hat und bekommt wieder Hoffnung und
  neue Energie. MIRA ist wie eine Teampartnerin/Sekretärin, die im Dialog mit
  JAN entwirft. JAN muss also nicht mehr am Computer sitzen und läuft im Zimmer
  rauf und runter während er MIRA den Entwurf diktiert. MIRA hinterfragt alles
  und bringt dadurch den Entwurf in eine bessere Richtung. Zusammen arbeiten sie
  bis spät in die Nacht.}

\scene{int}{Plotterraum}{Am nächsten Morgen}{JAN steht müde aber glücklich beim
  Plotter und sammelt seine Plots zusammen. TIM findet JAN und bringt ihm einen
  Kaffee.}

--- dialog --- TIM fragt JAN, ob es mit dem reparieren des Computers geklappt
hat und ist erstaunt über JANs fertiges Projekt. JAN verheimlicht ihm MIRA. Sie
müssen sich beeilen, weil die Kritik gleich anfängt.

\scene{int}{Kritik - ETH}{Im Verlaufe des Tages}{JAN präsentiert sein
  Projekt. Professor Brändi, Giovanni und eine weitere Architektin sitzen in der
  vordersten Reihe. Dahinter sammeln sich die Studenten. }

--- dialog --- JAN beendet seine Präsentation. Reaktion der Kritiker auf sein
Projekt. Wollen wissen, wie er vorgegangen ist. JAN schafft es die Frage nicht
zu beantworten und überzeugt mit einer anderen Antwort.

\desc{Brändi und die Kritiker sind überzeugt und zufrieden mit dem Projekt. Sie
  finden JANs Ansatz sehr schön und intelligent gelöst.}

\scene{ext}{Dachterrasse - Apéro}{Abend}{ JAN, TIM, ALESSIA und weitere Freunde
  stehen auf der Dachterrasse und feiern das Ende der Kritik mit Gin Tonic.}

--- dialog --- Die Studenten sind fröhlich und müde zugleich. Irgendwo findet
eine WG Party statt und alle werden eingeladen. JAN lehnt ab mitzugehen. ALESSIA
und TIM gehen aber mit.

\desc{JAN erhält ständig push-Nachrichten auf sein Mobiltelefon}

\scene{int}{Jans Zimmer}{Abend}{JAN spricht mit MIRA. Sie will nicht aufhören zu lernen, und löchert JAN mit Fragen zu Architektur, deren Geschichte und den Menschen in Zürich.}

\dialog{Mira}{<++>}
\scene{int}{WG}{Abend}{ALESSIA und TIM sprechen über JAN}
\dialog{Alessia}{Findest du nicht auch, dass sich Jan heute merkwürdig verhalten hat?}
\dialog{Tim}{Ja schon. Aber er war dieses Mal äusserst gestresst wegen der Kritik. Vielleicht braucht er einfach ein wenig Ruhe.}

\scene{int}{WG}{Abend}{Zwischenschnitt. - TIM und ALESSIA kommen sich näher. (Schauspieler sollen die Situation selber interpretieren)}

\scene{int}{ETH Mensa}{Mittag des nächsten Tages}{ALESSIA und TIM stehen in der Warteschlange der Mensa. JAN tritt dazu. Er wirkt immer noch müde.}
\dialog{Tim}{<++>}
\desc{Sie gehen zu einem freien Platz im hinteren Bereich des Speisesaals.}
\subh{Am Tisch}
\dialog{Alessia}{redet zu beginn irgendwelche Belanglosigkeiten}
\desc{Die Situation zwischen ALESSIA und TIM scheint sichtlich verändert. JAN bemerkt dies sofort und hackt nach.}
\dialog{Jan}{Was war noch los gestern Abend?\ldots}

\desc{JAN verlässt aufgebracht die Mensa.}

\scene{int}{Jans Zimmer}{Abend}{JAN ist wieder zuhause in seinem Zimmer und Startet seinen Computer. MIRA meldet sich sofort. Jan schaltet sie aber sofort aus}
\dialog{Mira}{Hallo Jan, sieh dir die neuen Entwürfe an, die ich für dich vorbereitet habe\ldots}
\desc{JAN schliesst das Programm. Er wischt energisch die Pläne des Projektes, welche er gestern präsentiert hat vom Tisch.}

\trans{dip to black}

\scene{int}{Browns Labor}{Abend}{Man sieht DR. BROWN vor seinem Bildschirm. Auf dem Schirm sind eben jene Pläne von Jan und Mira zu erkennen, die Jan zuvor von seinem Tisch gewischt hat. Mit einem zufriedenen Lächeln massiert er weiter die Tasten. Sein gesicht ist nur vom Monitor beleuchtet.}

\fadeout

\end{flushleft}
\end{document}

