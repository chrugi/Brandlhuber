\documentclass[a4paper,ngerman,11pt]{scrartcl}

\usepackage[utf8]{inputenc} \usepackage[T1]{fontenc} \usepackage{ngerman}

%\addtokomafont{dispositon}{\ttfamily}

\title{Episode 1} \subtitle{Genesis} \author{Sahra Roman \& Christian Sangvik}
\date{\today}

\newcommand{\linia}{\rule{.8\textwidth}{.1pt}}

\makeatletter

\renewcommand{\maketitle}{ \texttt{
\begin{center}
  \uppercase{\tiny{ ETH Zürich
    \enspace{}|\enspace{} D-ARCH
    \enspace{}|\enspace{} Professur Arno Brandlhuber
    \enspace{}|\enspace{} Herbstsemester 2017 }}\\ \linia{}\\
    \vspace*{5cm} \uppercase{\Huge{De Architectura}}\\
    \vspace*{2cm} \uppercase{\normalsize{von}}\\
    \vspace*{1cm} \uppercase{\LARGE{\MakeUppercase{\@author}}}\\
  \clearpage
    \vspace*{4cm} \textbf{\huge{\MakeUppercase{\@title}}}\\
    \vspace*{1cm}
    \Large{\textit{---\enspace{}\MakeUppercase{\@subtitle}\enspace{}---}}\\
    \vspace*{2cm}
\end{center}
}} \makeatother

\newcommand{\dialog}[2]{
\begin{center}
\uppercase{#1}\\
\vspace*{6pt}
\begin{minipage}{.5\textwidth}
{#2}\\
\end{minipage}\hfill
\end{center}
\vspace*{12pt}}

\newcommand{\scene}[4]{
\noindent{}\uppercase{#1}.\quad{}\uppercase{#2}\quad{}-\quad{}\uppercase{#3}\\

\noindent{{#4}\\}
\vspace*{24pt} }

\newcommand{\fadein}{
\noindent\uppercase{Fade in:}\\ }

\newcommand{\trans}[1]{ \hfill\uppercase{{#1}:}\\ }

\newcommand{\subh}[1]{
\noindent\uppercase{{#1}\\} }

\newcommand{\desc}[1]{\vspace*{12pt}
\noindent{}{#1}\\
\vspace*{12pt}}

\begin{document}

\maketitle

\fadein

\scene{int}{Zimmer JAN}{Morgen}{ JAN wacht auf und merkt, dass er verschlafen
  hat.  Er hat die Nacht an seinem Schreibtisch verbracht, wo er beim späten
  Arbeiten eingeschlafen ist. Im Stress packt er seine Sachen zusammen und eilt
  ohne Frühstück aus dem Haus.}

\scene{int}{Bus}{Morgen}{ JAN sitzt müde im Bus und schaut aus dem Fenster. Er
  sieht die Stadt Zürich von oben im morgendlichen Nebel.}

\scene{ext}{Koje - ETH}{Morgen}{ JAN trifft auf TIM. TIM wartet bereits rauchend
  mit einem Kaffee auf JAN vor dem HIL-Eingang.}

\scene{int}{Korridor - ETH}{Morgen}{ JAN und TIM gehen zusammen durch die
  Korridore in Richtung Koje.}

--- dialog --- Beziehung TIM-JAN soll klar werden. JANs Problem mit seinem Projekt
und Computer wird angesprochen. TIM soll ihm helfen (ist zuversichtlich)
Projektaufgabe soll auch klar werden.

\subh{ALESSIA kommt dazu}

--- dialog --- Veränderte Haltung von JAN zeigt sein Interesse and ALESSIA. TIM findet
es lustig.

\scene{int}{GUDZILLAs Büro}{Vormittag}{Dr. BROWN sitzt ungeduldig in GUDZILLAs
  Büro und wartet. GUDZILLA tritt ein.}

--- dialog --- GUDZILLA muss BROWN schlechte Nachrichten überbbringen. BROWN hat
etwas anderes erwartet.

\scene{int}{Koje - ETH}{Vormittag}{JAN sitzt frustriert vor seinem Computer. Das
  Programm Dreamcatcher stürzt immer wieder ab. Er bittet TIM um Hilfe, aber der
  kann ihm auch nicht helfen. Sie wenden sich an ALESSIA, da beide wissen wie gut
  sie mit Computern umgehen kann.}

--- dialog --- Schlussfolgerung ALESSIA, die Vorschlägt, dass er doch zum Helpdesk am
Nachmittag gehen soll.

\scene{int}{Helpdesk - ETH}{Nachmittag}{Dr. BROWN ist schlecht gelaunt. Er sitzt
  am Helpdesk und erklärt gerade einem STUDENTEN, dass er ihm nicht helfen
  kann. JAN wartet ab bis er an der Reihe ist und erklärt dann sein Problem.}

--- dialog --- JAN erklärt sein Problem, BROWN ist zuerst genervt, entwickelt aber
immer mehr Interesse für JANs Problem. Er muss den Laptop bis am Abend bei sich
behalten um das Problem zu beheben. BROWN macht ihm einen Gefallen und arbeitet
länger heute. Er wird JAN per Telefon benachrichtigen, wenn der Computer bereit
ist. JAN wird noch mehr frustriert, gibt sich dann aber geschlagen.

\scene{int}{Alumni Lounge - ETH}{Abend}{JAN sitzt frustriert und müde am
  Tisch. Neben sich einen leeren Kaffeebecher. Er schaut immer wieder auf sein
  Handy und wartet gestresst auf dein Anruf von BROWN. Gedankenverloren kritzelt
  er in ein Notizbuch. Endlich läutet das Telefon. Der Computer ist
  abholbereit. JAN packt hastig seine Sachen zusammen und geht.}

\scene{int}{Helpdesk - ETH}{Abend}{BROWN übergibt JAN den Laptop. JAN ist
  erleichtert.}

--- dialog --- BROWN hat die neue Version installiert. JAN ist glücklich.

\scene{int}{JANs Zimmer}{Nacht}{JAN startet seinen Computer und dann auch gleich
  das Programm. Man sieht ihm an, dass er eigentlich überhaupt keine Motivation
  mehr hat und lieber gleich schlafen gehen würde. Sobald das Programm
  gestartet ist, fängt MIRA an zu sprechen.}

\dialog{Mira}{Hallo, Ich bin Mira. Wie kann ich Ihnen helfen?}
--- dialog --- MIRA stellt sich vor, und fragt wie sie helfen kann.

\desc{JAN zögert zunächst. Er ist überfordert mit der neuen, viel selbstständigeren
Version von Dreamcatcher. Bald aber erkennt er, dass es doch noch eine Chance
für ein gutes Projekt hat und bekommt wieder Hoffnung und neue Energie. MIRA ist
wie eine Teampartnerin/Sekretärin, die im Dialog mit JAN entwirft. JAN muss also
nicht mehr am Computer sitzen und läuft im Zimmer rauf und runter während er
MIRA den Entwurf diktiert. MIRA hinterfragt alles und bringt dadurch den Entwurf
in eine bessere Richtung. Zusammen arbeiten sie bis spät in die Nacht.}

\scene{int}{Plotterraum}{Am nächsten Morgen}{JAN steht müde aber glücklich beim
  Plotter und sammelt seine Plots zusammen. TIM findet JAN und bringt ihm einen
  Kaffee.}

--- dialog --- TIM fragt JAN, ob es mit dem reparieren des Computers geklappt hat
und ist erstaunt über JANs fertiges Projekt. JAN verheimlicht ihm MIRA. Sie
müssen sich beeilen, weil die Kritik gleich anfängt.

\scene{int}{Kritik - ETH}{Im Verlaufe des Tages}{JAN präsentiert sein
  Projekt. Professor Brändi, Giovanni und eine weitere Architektin sitzen in der
  vordersten Reihe. Dahinter sammeln sich die Studenten. }

--- dialog --- JAN beendet seine Präsentation. Reaktion der Kritiker auf sein
Projekt. Wollen wissen, wie er vorgegangen ist. JAN schafft es die Frage nicht
zu beantworten und überzeugt mit einer anderen Antwort.

\desc{Brändi und die Kritiker sind überzeugt und zufrieden mit dem Projekt. Sie finden
JANs Ansatz sehr schön und intelligent gelöst.}

\scene{int}{Dachterrasse - Apero}{Abend}{ JAN, TIM, ALESSIA und weitere Freunde
  stehen auf der Dachterrasse und feiern das Ende der Kritik mit Gin Tonic.}

--- dialog --- Die Studenten sind fröhlich und müde zugleich. Irgendwo findet eine
WG Party statt und alle werden eingeladen. JAN lehnt ab mitzugehen. ALESSIA und TIM
gehen aber mit.

\end{document}

