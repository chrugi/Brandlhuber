\documentclass[a4paper,ngerman,11pt]{scrartcl}

  \usepackage[utf8]{inputenc}
  \usepackage[T1]{fontenc}
  \usepackage{ngerman}

  \title{Korridorgespräche}
  \subtitle{Jan und Tim}
  \author{Sahra Roman \and Christian Sangvik}
  \date{\today}

  \renewcommand{\familydefault}{\ttdefault}
  \renewcommand{\ttdefault}{pcr}

  \newcommand{\linia}{\rule{.8\textwidth}{.1pt}}

  \makeatletter
  \renewcommand{\maketitle}{
  \texttt{
    \pagestyle{empty}
    \begin{center}
      \uppercase{\tiny{
        ETH Zürich
        \enspace{}|\enspace{}
        D-ARCH
        \enspace{}|\enspace{}
        Professur Arno Brandlhuber
        \enspace{}|\enspace{}
        Herbstsemester 2017}}\\
      \linia{}\\
      \vfill
      \uppercase{\Huge{Scenario}}\\
      \vfill
      \uppercase{\normalsize{Drehbuch Szene}}\\
      \uppercase{\LARGE{\MakeUppercase{EP: I; Jan und Tim Korridor}}}\\
      \vspace*{1cm}
      \uppercase{\normalsize{von}}\\
      \uppercase{\LARGE{\MakeUppercase{Sahra Roman \&}}}\\
      \uppercase{\LARGE{\MakeUppercase{Christian Sangvik}}}\\
      \vspace*{.5cm}
      \uppercase{\normalsize{Version vom \MakeUppercase{\@date}}}
    \end{center}
  \clearpage
    {~}
  \clearpage
    \pagestyle{myheadings}
    \setcounter{page}{1}
    \begin{center}
      \vspace*{3cm}
      \huge{\MakeUppercase{\@title}}\\
      \vspace*{1cm}
      \Large{\textit{--\enspace{}\MakeUppercase{\@subtitle}\enspace{}--}}\\
      \vspace*{2cm}
    \end{center}
  }}
  \makeatother

  % C O M M A N D S

  % fade in
  \newcommand{\fadein}{\noindent\uppercase{Fade in:}\\}

  % fade out
  \newcommand{\fadeout}{
    \vspace*{12pt}
    \noindent\uppercase{Fade out:}\\}

  % pagebreak continued
  \newcommand{\pgbr}{
    \begin{center}
      \uppercase{(continued)}
    \end{center}
    \clearpage
    \begin{center}
      \uppercase{continued:}
    \end{center}
    \vspace*{12pt}}

  % transition
  \newcommand{\trns}[1]{
    \vspace*{12pt}
    \hfill\uppercase{{#1}:}\\}

  % subheader
  \newcommand{\sbh}[1]{
    \vspace*{12pt}
    \noindent\uppercase{{#1}\\}}

  % dialog normal
  \newcommand{\dlg}[2]{
    \begin{center}
      \uppercase{#1}\\
      \vspace*{6pt}
      \begin{minipage}{.8\textwidth}
        {#2}\\
    \end{minipage}\hfill
    \end{center}}

  % dialog extended, with parentheses
  \newcommand{\xdlg}[3]{
    \begin{center}
      \uppercase{#1}\\({#3})\\
      \vspace*{6pt}
      \begin{minipage}{.8\textwidth}
        {#2}\\
      \end{minipage}\hfill
    \end{center}}

  % description of scene
  \newcommand{\dsc}[1]{
    \vspace*{12pt}
    \noindent{}{#1}\\}

  % scene
  \newcommand{\scn}[4]{
    \vspace*{12pt}
    \noindent{\uppercase{{#1}.\quad{}{#2}\quad{}-\quad{}{#3}}}\\
    \dsc{#4}}

  % uppercase
  \newcommand{\scnn}[1]{\uppercase{#1}}


% ############################################################################ %
% ------------------------------ D O C U M E N T ----------------------------- %
% ############################################################################ %

\begin{document}
\begin{flushleft}

\maketitle

% \fadein  - fade in, used at beginning of script, \fadein
% \fadeout - fade out, used at end of script, \fadeout
% \pgbr    - Page break. Break page at that point and mark both ends with
%            "continued". Use with caution - other page sizes could break at
%            different points. This could give unwanted results, \pgbr
% \trns    - transition between shots, \trns{TRANSITION}
%            (\trns{dip to black})
% \sbh     - subheader in scene, \sbh{SUBHEADER}
%            (\sbh{Character walks in})
% \dlg     - dialog, \dlg{NAME}{SENTENCE}
%            (\dlg{Tim}{Hi!})
% \xdlg    - extended dialog, \xdlg{NAME}{SENTENCE}{EXTENSION}
%            (\dlg{Tim}{What?!}{excited})
% \scn     - scene, \scn{INT/EXT}{LOCATION}{TIME}{DESCRIPTION}
%            (\scn{int}{at home}{morning}{Tim wakes up in his bed.})
% \dsc     - scene description, \dsc{DESCRIPTION}
%            (\dsc{Tim still looks tired. He brushes his teeth})
% \scnn    - actor name in scene. This is put in the scene description to mark
%            a new character in a scene, \scnn{NAME}
%            (\scn{int}{at work}{morning}{\scnn{Tim} sits bored at his desk})

% ---------- S C R I P T ----------------------------------------------------- %

\fadein

\scn{int}{Korridor - Hochschule}{Morgen - Onetake}{\scnn{Jan} und \scnn{Tim}
  gehen zusammen durch die Korridore in Richtung Koje.}

\xdlg{Jan}{Ich bin gestern am Schreibtisch eingeschlafen. Ich komme einfach
  nicht weiter mit meinem Projekt.\\ Danke für deine Hilfe.}{gestresst}

\xdlg{Tim}{Kein Problem. Wir kennen es ja nicht anders von dir.}{grinsend}

\dlg{Jan}{Also, wie hast du den Cluster an der Langstrasse gelöst?}

\dlg{Tim}{Du musst den Verkehrsparameter sehr schwer gewichten. Sonst hast du am
  Schluss eine Fussgängerzone.}

\dlg{Jan}{Wenn ich das so mache, bekomme ich immer nur Fehlermeldungen.}

\fadeout

\end{flushleft}
\end{document}
