\documentclass[a4paper,ngerman,11pt]{scrartcl}

  \usepackage[utf8]{inputenc}
  \usepackage[T1]{fontenc}
  \usepackage{ngerman}

  \title{Giovanni trifft Präsidentin Fluri}
  \subtitle{Giovanni beklagt sich}
  \author{Sahra Roman \and Christian Sangvik}
  \date{\today}

  \renewcommand{\familydefault}{\ttdefault}
  \renewcommand{\ttdefault}{pcr}

  \newcommand{\linia}{\rule{.8\textwidth}{.1pt}}

  \makeatletter
  \renewcommand{\maketitle}{
  \texttt{
    \pagestyle{empty}
    \begin{center}
      \uppercase{\tiny{
        ETH Zürich
        \enspace{}|\enspace{}
        D-ARCH
        \enspace{}|\enspace{}
        Professur Arno Brandlhuber
        \enspace{}|\enspace{}
        Herbstsemester 2017}}\\
      \linia{}\\
      \vfill
      \uppercase{\Huge{Scenario}}\\
      \vfill
      \uppercase{\normalsize{Drehbuch Szene}}\\
      \uppercase{\LARGE{\MakeUppercase{EP: IV; Giovanni und Fluri}}}\\
      \vspace*{1cm}
      \uppercase{\normalsize{von}}\\
      \uppercase{\LARGE{\MakeUppercase{Sahra Roman \&}}}\\
      \uppercase{\LARGE{\MakeUppercase{Christian Sangvik}}}\\
      \vspace*{.5cm}
      \uppercase{\normalsize{Version vom \MakeUppercase{\@date}}}
    \end{center}
  \clearpage
    {~}
  \clearpage
    \pagestyle{myheadings}
    \setcounter{page}{1}
    \begin{center}
      \vspace*{3cm}
      \huge{\MakeUppercase{\@title}}\\
      \vspace*{1cm}
      \Large{\textit{--\enspace{}\MakeUppercase{\@subtitle}\enspace{}--}}\\
      \vspace*{2cm}
    \end{center}
  }}
  \makeatother

  % C O M M A N D S

  % fade in
  \newcommand{\fadein}{\noindent\uppercase{Fade in:}\\}

  % fade out
  \newcommand{\fadeout}{
    \vspace*{12pt}
    \noindent\uppercase{Fade out:}\\}

  % transition
  \newcommand{\trns}[1]{
    \vspace*{12pt}
    \hfill\uppercase{{#1}:}\\}

  % subheader
  \newcommand{\sbh}[1]{
    \vspace*{12pt}
    \noindent\uppercase{{#1}\\}}

  % dialog normal
  \newcommand{\dlg}[2]{
    \begin{center}
      \uppercase{#1}\\
      \vspace*{6pt}
      \begin{minipage}{.8\textwidth}
        {#2}\\
    \end{minipage}\hfill
    \end{center}}

  % dialog extended, with parentheses
  \newcommand{\xdlg}[3]{
    \begin{center}
      \uppercase{#1}\\({#3})\\
      \vspace*{6pt}
      \begin{minipage}{.8\textwidth}
        {#2}\\
      \end{minipage}\hfill
    \end{center}}

  % description of scene
  \newcommand{\dsc}[1]{
    \vspace*{12pt}
    \noindent{}{#1}\\}

  % scene
  \newcommand{\scn}[4]{
    \vspace*{12pt}
    \noindent{\uppercase{{#1}.\quad{}{#2}\quad{}-\quad{}{#3}}}\\
    \dsc{#4}}

  % uppercase
  \newcommand{\scnn}[1]{\uppercase{#1}}


% ############################################################################ %
% ------------------------------ D O C U M E N T ----------------------------- %
% ############################################################################ %

\begin{document}
\begin{flushleft}

\maketitle

% \fadein  - fade in, used at beginning of script, \fadein
% \fadeout - fade out, used at end of script, \fadeout
% \trns    - transition between shots, \trns{TRANSITION}
%            (\trns{dip to black})
% \sbh     - subheader in scene, \sbh{SUBHEADER}
%            (\sbh{Character walks in})
% \dlg     - dialog, \dlg{NAME}{SENTENCE}
%            (\dlg{Tim}{Hi!})
% \xdlg    - extended dialog, \xdlg{NAME}{SENTENCE}{EXTENSION}
%            (\dlg{Tim}{What?!}{excited})
% \scn     - scene, \scn{INT/EXT}{LOCATION}{TIME}{DESCRIPTION}
%            (\scn{int}{at home}{morning}{Tim wakes up in his bed.})
% \dsc     - scene description, \dsc{DESCRIPTION}
%            (\dsc{Tim still looks tired. He brushes his teeth})
% \scnn    - actor name in scene. This is put in the scene description to mark
%            a new character in a scene, \scnn{NAME}
%            (\scn{int}{at work}{morning}{\scnn{Tim} sits bored at his desk})

% ---------- S C R I P T ----------------------------------------------------- %

\fadein

\scn{int}{Büro von Stadtpräsidentin Fluri}{Morgen}{\scnn{Stadtpräsidentin Fluri}
  sitzt an ihrem Schreibtisch im Büro. Es klopft. Sie bittet den Besucher
  hinein.}

\dlg{Fluri}{Herein!}

\dsc{\scnn{Giovanni Benini} tritt ein. Er sieht aufgebracht und frustriert
  aus. Energisch geht er auf Fluri zu. Fluri ist überrascht, da sie nicht mit
  einem ausserordentlichen Besuch von Giovanni gerechnet hat.}

\xdlg{Giovanni}{Entschuldigen Sie die Störung, aber ich muss mit Ihnen über die
  untragbaren Zustände im Stadtbauamt sprechen.}{hastig und enerviert}

\xdlg{Fluri}{Was ist denn los in Ihrem Amt?}{ruhig, gefasst}

\dlg{Giovanni}{Keiner unserer Mitarbeiter kann mehr auf unsere eigene
  Infrastruktur zugreifen. Alles ist blockiert. Kein Datenzugriff auf dem
  Server, nicht einmal der Drucker geht.\\Wir haben natürlich versucht intern
  den Fehler zu finden. Aber nichteinmal unser eigener Informatikbeauftragte hat
  Zugriff auf irgendetwas. Er sagt, die Sperre kommt von weiter oben, was mich
  hierher führt.\\Was ist hier eigentlich los?}

\xdlg{Fluri}{Davon weiss ich nichts. Seit wann liegt dieses Problem
  vor?}{aufgesetzt empört}

\dlg{Giovanni}{Verstehen sie? Wir können so nicht arbeiten. Wir können GAR
  NICHTS tun. So kann man das Amt gleich weglassen!}

\dlg{Fluri}{Ich habe in zwanzig Minuten eine wichtige Sitzung mit dem
  Stadtrat. Ich werde dem aber so schnell wie möglich nachgehen. Ich melde mich
  sobald ich etwas weiss.}

\dsc{Fluri deutet zur Tür. Giovanni zögert. Dann geht er aber ungläubig,
  kopfschüttelnd und einen Fluch unterdrückend zum Ausgang und verlässt den
  Raum.\\Fluri bleibt an ihrem Schreibtisch sitzen und beginnt zu grinsen. Sie
  wittert eine gute Möglichkeit Giovannis Befürchtung in die Tat umzusetzen, und
  so ihr Budget zu straffen, was für ihre Wahl nur zuträglich sein kann.}


\fadeout

\end{flushleft}
\end{document}
