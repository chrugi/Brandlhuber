\documentclass[a4paper,ngerman,11pt]{scrartcl}

  \usepackage[utf8]{inputenc}
  \usepackage[T1]{fontenc}
  \usepackage{ngerman}

  \title{Episode I}
  \subtitle{Genesis}
  \author{Sahra Roman \and Christian Sangvik}
  \date{\today}

  \renewcommand{\familydefault}{\ttdefault}
  \renewcommand{\ttdefault}{pcr}

  \newcommand{\linia}{\rule{.8\textwidth}{.1pt}}

  \makeatletter
  \renewcommand{\maketitle}{
  \texttt{
    \pagestyle{empty}
    \begin{center}
      \uppercase{\tiny{
        ETH Zürich
        \enspace{}|\enspace{}
        D-ARCH
        \enspace{}|\enspace{}
        Professur Arno Brandlhuber
        \enspace{}|\enspace{}
        Herbstsemester 2017}}\\
      \linia{}\\
      \vfill
      \uppercase{\Huge{Scenario}}\\
      \vfill
      \uppercase{\normalsize{Drehbuch}}\\
      \uppercase{\LARGE{\MakeUppercase{Episode I}}}\\
      \vspace*{1cm}
      \uppercase{\normalsize{von}}\\
      \uppercase{\LARGE{\MakeUppercase{Sahra Roman \&}}}\\
      \uppercase{\LARGE{\MakeUppercase{Christian Sangvik}}}\\
      \vspace*{.5cm}
      \uppercase{\normalsize{Version vom \MakeUppercase{\@date}}}
    \end{center}
  \clearpage
    {~}
  \clearpage
    \pagestyle{myheadings}
    \setcounter{page}{1}
    \begin{center}
      \vspace*{3cm}
      \huge{\MakeUppercase{\@title}}\\
      \vspace*{1cm}
      \Large{\textit{--\enspace{}\MakeUppercase{\@subtitle}\enspace{}--}}\\
      \vspace*{2cm}
    \end{center}
  }}
  \makeatother

  % C O M M A N D S

  % fade in
  \newcommand{\fadein}{\noindent\uppercase{Fade in:}\\}

  % fade out
  \newcommand{\fadeout}{
    \vspace*{12pt}
    \noindent\uppercase{Fade out:}\\}

  % transition
  \newcommand{\trns}[1]{
    \vspace*{12pt}
    \hfill\uppercase{{#1}:}\\}

  % subheader
  \newcommand{\sbh}[1]{
    \vspace*{12pt}
    \noindent\uppercase{{#1}\\}}

  % dialog normal
  \newcommand{\dlg}[2]{
    \begin{center}
      \uppercase{#1}\\
      \vspace*{6pt}
      \begin{minipage}{.8\textwidth}
        {#2}\\
    \end{minipage}\hfill
    \end{center}}

  % dialog extended, with parentheses
  \newcommand{\xdlg}[3]{
    \begin{center}
      \uppercase{#1}\\({#3})\\
      \vspace*{6pt}
      \begin{minipage}{.8\textwidth}
        {#2}\\
      \end{minipage}\hfill
    \end{center}}

  % description of scene
  \newcommand{\dsc}[1]{
    \vspace*{12pt}
    \noindent{}{#1}\\}

  % scene
  \newcommand{\scn}[4]{
    \vspace*{12pt}
    \noindent{\uppercase{{#1}.\quad{}{#2}\quad{}-\quad{}{#3}}}\\
    \dsc{#4}}


% ############################################################################ %
% ------------------------------ D O C U M E N T ----------------------------- %
% ############################################################################ %

\begin{document}
\begin{flushleft}

\maketitle

\fadein

% ---------- scene ----------------------------------------------------------- %
% ############################################################################ %
\scn{int}{Jans Zimmer}{Morgen}{JAN AEBERSOLD wacht auf und merkt, dass er
  verschlafen hat.  Er hat die Nacht an seinem Schreibtisch verbracht, wo er
  beim späten Arbeiten eingeschlafen ist. Im Stress packt er seine Sachen
  zusammen und eilt ohne Frühstück aus dem Haus.}


% ---------- scene ----------------------------------------------------------- %
% ############################################################################ %
\scn{int}{Bus}{Morgen}{JAN sitzt müde im Bus und schaut aus dem Fenster. Er
  sieht die Stadt Zürich von oben im morgendlichen Nebel.}


% ---------- scene ----------------------------------------------------------- %
% ############################################################################ %
\scn{ext}{ETH Hönggerberg}{Morgen}{JAN trifft auf TIM BERGMANN. Tim wartet
  bereits vor dem Haupteingang des HIL Gebäudes. Er raucht eine Zigarette und
  hält einen Becher Kaffee in der Hand. Als Jan Tim erreicht, betreten sie
  schnell das Gebäude.}

\xdlg{Tim}{Na schau mal an, wer da kommt! Hast du's auch noch
  geschafft?}{grinsend}

\xdlg{Jan}{Du hättest mir lieber auch einen Kaffee mitgebracht.}{murmelnd}

\dsc{TIM gibt JAN grinsend den zweiten Becher Kaffee, den er versteckt gehalten
  hat. Sie betreten das Gebäude.}


% ---------- scene ----------------------------------------------------------- %
% ############################################################################ %
\scn{int}{Korridor - ETH}{Morgen}{Onetake - JAN und TIM gehen zusammen durch die
  Korridore in Richtung Koje.}

\dlg{Jan}{Ich bin gestern am Schreibtisch eingeschlafen. Ich komme einfach nicht
  weiter mit meinem Projekt.\\ Vielen Dank, dass du mir dabei hilfst.}

\dlg{Tim}{Kein Problem. Ich habe ja sonst nichts zu tun.}

\dlg{Jan}{Shut up, Bitch!}

\dsc{Sie erreichen die Koje. Nach einer kurzen Pause}

\dlg{Jan}{Wie hast du den Cluster an der Langstrasse gelöst?}

\dlg{Tim}{Du musst den Verkehrsparameter sehr schwer gewichten. Sonst hast du am
  Schluss eine Fussgängerzone.}

\dlg{Jan}{Das wäre doch auch nicht so schlecht. Aber es löst das Problem der
  öffentlichen Anbindung nicht.}

\sbh{ALESSIA kommt dazu}

\dlg{ALlessia}{Guten Morgen ihr zwei! Naja mindestens dich, Jan, brauche ich
  wohl kaum zu fragen, ob du gut geschlafen hast. Kommst du voran?}

\dlg{Jan}{Tim ist meine letzte Hoffnung. Ich habe schon so zu wenig Zeit. Wenn
  das jetzt nicht klappt, brauche ich morgen erst gar nicht zu kommen.}

\dlg{Alessia}{Dann will ich euch nicht länger aufhalten. Viel Erfolg!}

\dsc{ALESSIA geht und setzt sich an ihren eigenen Platz, wo sie den Laptop
  aufklappt. JAN schaut ihr abwesend hinterher.}

\xdlg{Tim}{So! Jetzt fertig gesabbert! Weiter an deinem Projekt.}{lachend}

\trns{hard cut}


% ---------- scene ----------------------------------------------------------- %
% ############################################################################ %
%\scn{int}{GUDZILLAs Büro}{Vormittag}{DR. STANISLAV BROWN sitzt ungeduldig im
%  Büro von ETH PRÄSIDENT LINO GUDZILLA und wartet. Gudzilla tritt ein.}
%
%\dlg{Gudzilla}{Dr. Brown. Vielen Dank, dass sie so kurzfristig kommen konnten.}
%
%\dsc{Gudzilla setzt sich}
%
%\dlg{Gudzilla}{Nun, ich will ganz offen mit Ihnen sprechen. Sie wissen ja
%  sicherlich, dass die Stadt Zürich die Gelder für die Forschung und Bildung
%  gekürzt hat. Auch wenn wir immernoch den Löwenanteil davon haben, sind wir
%  nichtsdestotrotz betroffen. Sie verstehen sicherlich, dass ich meine
%  Forschungsgelder sehr sparsam einsetzen muss.}
%
%\dlg{Brown}{Ja, das ist mir zu Ohren gekommen. Aber was hat dies mit mir zu
%  tun?}
%
%\dsc{GUDZILLA wartet einen Moment, bevor er fortfährt}
%
%\dlg{Gudzilla}{Ihre Forschung hat bis jetzt sehr wenige Früchte getragen. Ich
%  habe Ihnen bereits mehrfach einen Zeitaufschub gewährt um zu Resultaten zu
%  kommen. Aber Sie sind Ihren Teil mit den Erfolgen bis jetzt schuldig
%  geblieben. Sie werden verstehen, dass ich unter diesen Umständen Ihr Projekt
%  nicht mehr länger finanzieren kann.}
%
%\dlg{Brown}{Aber ich bin nun endlich unmittelbar vor dem grossen Durchbruch!
%  Das Programm funktioniert und ist ganz kurz vor der Vollendung!  Geben Sie mir
%  doch nur noch eine Wo..}
%
%\dsc{GUDZILLA unterbricht Brown}
%
%\dlg{Gudzilla}{Ich habe mich doch bezüglich den zeitlichen Rahmenbedingungen das
%  letzte Mal klar und deutlich ausgedrückt, oder? Es tut mir leid, aber ich kann
%  hier keine Ausnahme machen. Ich kann nichts mehr für sie tun.}
%
%\dlg{Brown}{Bin ich... Entlassen?}
%
%\dlg{Gudzilla}{Sie sind an ihrem Departement ja als Informatiker
%  eingestellt. Daran ändert sich nichts. Aber als wissenschaftlichen Mitarbeiter
%  muss ich sie leider freistellen.}
%
%\trns{hard cut}


% ---------- scene ----------------------------------------------------------- %
% ############################################################################ %
\scn{int}{Koje - ETH}{Vormittag}{JAN sitzt frustriert vor seinem Computer. Das
  Programm Dreamfetcher stürzt immer wieder ab. Er bittet TIM um Hilfe, aber der
  kann ihm auch nicht helfen. Sie wenden sich an ALESSIA, da beide wissen wie
  gut sie mit Computern umgehen kann.}

\xdlg{Jan}{Verfluchte Dreckskacke!}{energisch, aufgebracht}

\dlg{Tim}{Was hast du denn jetzt wieder angestellt?}

\dlg{Jan}{Ich habe keinen Nerv mehr für diesen Scheiss!}

\dlg{Tim}{Was ist denn los?}

\dlg{Jan}{Das Programm stürzt jedes mal ab, wenn ich die Gewichtung der
  Parameter ändere.}

\dlg{Tim}{Du solltest dir vielleicht endlich einen neuen Computer
  zulegen... Zeig mal her!}

\dsc{Jan gibt Tim seinen Laptop. Sie probieren beide ein wenig unbeholfen herum
  das Problem zo lösen, ohne jedoch Erfolg zu haben.}

\dlg{Tim}{Hmm... Keine Ahnung. Fragen wir Ale, sie versteht sich besser
  darauf. Ale! Hast du kurz zeit?}

\xdlg{Alessia}{Um was geht's denn?}{überrascht}

\dlg{Tim}{Jan hat ein Problem mit Dreamfetcher, das ich nicht lösen kann. Kannst
  du dir das mal ansehen?}

\dsc{ALESSIA tritt zu Jans Computer und beginnt sofort zu tippen. Jan und Tim
  schauen gespannt zu.}

\xdlg{Alessia}{So. Nach einem Neustart müsste es funktionieren.}{überzeugt}

\dsc{Jan startet den Computer neu und gibt sein Passwort ein.}

\xdlg{Alessia}{Ehrlich? Password.1234?}{augenrollend}

\dsc{Jan erwidert verlegen Alessias Blick. Er startet das Programm, welches
  sofort wieder abstürzt.}

\xdlg{Alessia}{Hmm... Das verstehe ich jetzt nicht. Vielleicht gehst du doch
  lieber zum Helpdesk.}{verlegen}


% ---------- scene ----------------------------------------------------------- %
% ############################################################################ %
\scn{int}{Helpdesk - ETH}{Nachmittag}{DR. STANISLAV BROWN ist schlecht
  gelaunt. Er sitzt am Helpdesk und erklärt gerade einem STUDENTEN, dass er ihm
  nicht helfen kann. JAN wartet ab bis er an der Reihe ist und erklärt dann sein
  Problem.}

\dlg{Jan}{Hallo. Ich habe ein Problem mit `Dreamfetcher'. Das Programm stürzt
  immer ab, wenn ich versuche die Parametergewichtung zu ändern.}

\xdlg{Brown}{Hast du versucht den Computer neu zu starten?}{sichtlich genervt}

\dlg{Jan}{Natürlich. Aber das hat auch nichts geholfen. Ich glaube, es handelt
  sich um einen Softwarefehler. Vielleicht ist es auch wegen meinem alten
  Computer}

\dsc{Jan zeigt seinen alten Computer. Brown scheint offenbar interessierter.}

\dlg{Brown}{Zeig mal her. Ich werde mir das ansehen.}

\dsc{Brown tippt ein wenig in den Tasten herum.}

\dlg{Brown}{Hm. Ich fürchte, ich kann da nichts machen. Es scheint tatsächlich
  ein grundlegenderes Problem zu sein. Ich kann es mir höchstens im Detail
  ansehen und sehen, was ich machen kann. Aber ich fürchte, du musst deinen
  Computer bis heute Abend hier lassen.}

\dlg{Jan}{Aber das geht nicht. Ich habe morgen eine Kritik und muss noch viel
  arbeiten.}

\dlg{Brown}{Dann kann ich dir nicht helfen. Ich verstehe deine Not. Aber alles
  was ich machen kann, ist dir anzubieten, dass ich heute länger arbeite und
  versuche deinen Laptop wieder zum laufen zu bringen.}

\dsc{Jan scheint entäuscht, dass es keine einfache Lösung für sein Problem zu
  geben scheint. Als er an seine Abgabe denkt, haucht ein Anflug von Panik über
  sein gesicht. Da alles nichts zu nützen scheint, willigt er resigniert ein
  seinen Laptop da zu lassen.}

\dlg{Jan}{Also gut. Wenn sonst nichts geht, müssen wir es eben so versuchen.}

\dlg{Brown}{Keine Sorge. Ich werde deinen Computer bis heute abend fertig
  haben. Ich rufe dich an, wenn ich so weit bin.}

\dsc{Jan hinterlässt seine Telefonnummer auf einem Zettel und geht davon.}


% ---------- scene ----------------------------------------------------------- %
% ############################################################################ %
\scn{int}{Alumni Lounge - ETH}{Abend}{JAN sitzt frustriert und müde am
  Tisch. Neben sich einen leeren Kaffeebecher. Er schaut immer wieder auf sein
  Mobiltelefon und wartet gestresst auf den Anruf von BROWN. Gedankenverloren
  kritzelt er in ein Notizbuch und versucht sein Projekt von Hand zu
  skizzieren. Endlich klingelt das Telefon. Sein Computer ist abholbereit. JAN
  packt hastig seine Sachen zusammen und geht.}

\xdlg{Brown}{Hallo Jan.\\ Ich habe deinen Computer nun fertig. Du kannst ihn nun
  abholen.}{über das Telefon}

\dlg{Jan}{Super ich komme sofort!}


% ---------- scene ----------------------------------------------------------- %
% ############################################################################ %
\scn{int}{Helpdesk - ETH}{Abend}{BROWN übergibt JAN den Laptop. JAN ist
  sichtlich erleichtert.}

\dlg{Brown}{Mit der alten Installation war nichts mehr anzufangen. Ich habe dir
  deshalb die neueste Version installiert. Mit dieser hast du nun auch nicht
  mehr das Problem mit den Lizenzen.}

\xdlg{Jan}{Vielen Dank! Ich wollte schon aufgeben.\\ Aber unter Umständen muss
  ich das nun auch. Bis morgen bleibt nicht mehr viel Zeit, und es ist noch eine
  Menge zu tun.}{Glücklich, aber trotzdem mit \\ einem Anflug von Resignation}

\dlg{Brown}{Mach dir deswegen mal keine allzu grossen Gedanken. Du wirst
  feststellen, dass die neue Version um einiges mächtiger ist als die alte. Die
  Löst dir alle Probleme quasi von selbst.}

\xdlg{Jan}{Dann werde ich sie wohl gleich mal ausprobieren und an ihre Grenze
  bringen. Schönen Abend noch, und nochmals vielen Dank!}{ungläubig}

\xdlg{Brown}{Ich glaube nicht, dass du das schaffst. Das wünsche ich dir
  auch.}{lachend}

\trns{dissolve to}


% ---------- scene ----------------------------------------------------------- %
% ############################################################################ %
\scn{int}{JANs Zimmer}{Nacht}{JAN startet seinen Computer und dann auch gleich
  das Programm. Man sieht ihm an, dass er eigentlich überhaupt keine Motivation
  mehr hat und lieber gleich schlafen gehen würde. Sobald das Programm gestartet
  ist, fängt MIRA an zu sprechen.}

\dlg{Mira}{Hallo! Ich bin Mira, Ihre neue Assistentin für den architektonischen
  Entwurf. Wie kann ich Ihnen helfen?}

\dsc{Jan nimmt die Maus in die Hand und versucht in der neuen Oberfläche einige
  Parameter einzustellen.}

\dlg{Mira}{Sie können wie gewohnt Ihre Angaben im klassischen Interface
  einstellen. Eine effizientere Methode wäre jedoch mit dem neuen
  Sprachinterface.  Möchten Sie die Sprachsteuerung für Ihr Projekt starten?}

\xdlg{Jan}{Ja ja...}{ungläubig}

\dlg{Mira}{Die Sprachsteuerung ist nun aktiviert.\\ Wo soll Ihr neues Projekt
  liegen?}

\dsc{Jan zögert zunächst. Er ist überfordert mit der neuen, viel
  selbstständigeren Version von Dreamfetcher. Bald aber erkennt er, dass es doch
  noch eine Chance für ein gutes Projekt hat und bekommt wieder Hoffnung und
  neue Energie. MIRA ist wie eine Teampartnerin oder Sekretärin, die im Dialog
  mit Jan entwirft. Jan muss also nicht mehr am Computer sitzen und läuft im
  Zimmer rauf und runter während er Mira den Entwurf diktiert. Mira hinterfragt
  alles und bringt dadurch den Entwurf in eine bessere Richtung. Zusammen
  arbeiten sie bis spät in die Nacht.}


% ---------- scene ----------------------------------------------------------- %
% ############################################################################ %
\scn{int}{Plotterraum}{Am nächsten Morgen}{JAN steht müde aber glücklich und
  erleichtert beim Plotter und sammelt seine Ausdrucke zusammen. TIM findet Jan
  und bringt ihm einen Kaffee.}

\dlg{Tim}{Hat es also noch geklappt mit deinem Computer gestern? Brändi wird es
  sicherlich verstehen, dass du unter diesen Umständen den Entwurf nicht
  vollenden konntest.}

\xdlg{Jan}{Ach, weisst du, ich bin heute früh fertig geworden}{grinsend}

\xdlg{Tim}{Zeig mal her!}{überrascht}

\dsc{Tim schaut sich Jans Ausdrucke ein wenig näher an.}

\dlg{Tim}{Diese Pläne hier sind richtig gut. Und wie ich sehe, hattest du sogar
  die Zeit, eine schöne Visualisierung anzufertigen. Wie hast du das nur
  gemacht?}

\xdlg{Jan}{Ich habe die ganze Nacht daran gesessen. Offenbar zahlt sich
  beharrlichkeit und fleiss im letzten Moment durchaus aus}{ausweichend}

\dsc{Als Jan seine Ausdrucke zusammen und zugeschnitten hat, schaut er auf die
  Uhr.}

\dlg{Jan}{Wenn wir vor der Kritik noch eine Zigarette rauchen wollen, müssen wir
  JETZT gehen.}

\dsc{Die beiden nehmen ihre jeweiligen Pläne unter den Arm und brechen auf.}


% ---------- scene ----------------------------------------------------------- %
% ############################################################################ %
\scn{int}{Koje bei der Kritik - ETH}{Im Verlaufe des Tages}{JAN präsentiert sein
  Projekt. PROFESSOR BRÄNDI, GIOVANNI BENINI und eine weitere Architektin sitzen
  in der vordersten Reihe. Dahinter sammeln sich die Studenten. Jan beendet
  gerade seine Präsentation.}

\xdlg{Jan}{Und so, glaube ich, kann die Lebendigkeit dieses Stadtteils bewahrt
  werden. Wenn man auf die neuen Bedürfnisse eingeht, aber die alten
  bewahrt}{sichtlich nervös}

\dlg{Prof. Brändi}{Sie zeigen hier eine Reihe sehr interessanter Ansäzte.  IMich
  würde interessieren, wie sie dabei vorgegangen sind. Auch da ich mich nicht
  erinnern kann diese Richtung in Ihrem bisherigen Prozess gelesen zu haben.
  Verstehen sie mich nicht falsch, ich finde die Richtung nicht schlecht, aber
  wie sind sie zu Ihrem Sinneswandel gekommen?}

\xdlg{Jan}{Ich war mit meinem bisherigen Ansatz nicht mehr zu frieden, und habe
  alles komplett von vorne aufgerollt.}{Nicht lügend, aber die Wahrheit
  verbergend}

\dlg{Giovanni}{Wie sind Sie demnach bei Ihrem neuen Ansatz darauf gekommen, die
  Verteilung der Öffentlichkeit und der Erschliessung auf diese Weise zu
  lösen. Welche Parameter waren Ihnen persönlich so wichtig, dass Sie auf dieses
  Resultat kommen?}

\xdlg{Jan}{Nun, ich will nicht allzu sehr ins Detail meiner Parameter gehen. Mir
  war es aber wichtig die Ausgewogenheit von Bestand und der integration vom
  Neuen zu erreichen. Sonst wäre meine Idee von Grund auf nicht
  aufgegangen.}{verlegen und suchend}

\xdlg{Architektin}{Ja, ich weiss nicht richtig, was zu sagen.\\ Mir gefällt
  es.\\ Können Sie mir nochmals die Aufteilung im Erdgeschoss
  aufzeigen?}{kopfnickend}

\dsc{Brändi und die Kritiker sind überzeugt und zufrieden mit dem Projekt. Sie
  finden Jans Ansatz sehr schön und intelligent gelöst.}


% ---------- scene ----------------------------------------------------------- %
% ############################################################################ %
\scn{ext}{Dachterrasse - Apéro}{Abend}{ JAN, TIM, ALESSIA und weitere Freunde
  stehen auf der Dachterrasse und feiern das Ende der Kritik mit einem oder zwei
  Gin Tonic.}

\dlg{Alessia}{So, das hätten wir geschafft. Nun haben wir endlich einen Tag
  Pause.}

\xdlg{Tim}{Du sagst es. Es war ein harter kampf bis vorgestern Abend, als ich
  fertig wurde.}{grinsend}

\dlg{Jan}{Mensch bin ich froh, dass es vorüber ist. Aber es ist ja zum glück
  recht gut gelaufen.}

\dlg{Tim}{Ja Jan. Wie hast du das geschafft? Ich meine, als ich mich gestern mit
  dir zusammengesetzt habe, war dein Projekt noch nirgends.. Und heute?  Heute
  überflügelst du mich.}

\dsc{Tim boxt Jan freundschaftlich an die Schulter.}

\dlg{Alessia}{Wollen wir nachher noch feiern gehen? Bei Jens findet heute noch
  eine kleine Feier statt.}

\dsc{JENS geht zufällig gerade an der Gruppe vorbei.}

\dlg{Jens}{Ja stimmt. Es wäre schön, wenn ihr alle mitkommt. Wir brechen in
  einer halben Stunde auf. Betrachten wir dieses Apero als Vorgeschmack.\\
  Übrigens, gratuliere Jan. Dein Projekt war hammerhart.}

\dlg{Jan}{Vielen Dank. Aber ich werde nicht mitgehen. Ich bin nach heute Nacht
  hundemüde und will nur noch in mein Bett und schlafen.}

\dlg{Alessia}{Schade. Aber ich und Tim kommen gerne. oder?}

\dlg{Tim}{Ja sicher! Eine Fete in Jens' WG lasse ich mir doch nicht entgehen.}

\dsc{Jan erhält ständig push-Nachrichten auf sein Mobiltelefon und ist häufig
  geistesabwesend. Er verabschiedet sich und geht alleine in Richtung Bus.}


% ---------- scene ----------------------------------------------------------- %
% ############################################################################ %
\scn{int}{Jans Zimmer}{Abend}{JAN spricht mit MIRA. Sie will nicht aufhören zu
  lernen, und löchert JAN mit Fragen zu Architektur, deren Geschichte und den
  Menschen in Zürich.}

\dlg{Mira}{Welches Design würden Sie persönlich für eine öffentliche Bibliothek
  bevorzugen?}

\dsc{Mira zeigt Jan auf dessen Monitor drei verschiedene Bilder die allesamt
  unterschiedliche Gebäudearten aufzeigen.}

\dlg{Jan}{Das Haus rechts.}

\dlg{Mira}{Würden die Menschen in Zürich ein solches Gebäude als Schule
  akzeptieren?}

\dsc{Mira zeigt Jan ein anderes Bild eines Gebäudes.}


% ---------- scene ----------------------------------------------------------- %
% ############################################################################ %
\scn{int}{WG}{Abend}{ALESSIA und TIM sprechen über Jan}

\dlg{Alessia}{Findest du nicht auch, dass sich Jan heute sehr merkwürdig
  verhalten hat?}

\dlg{Tim}{Ja schon. Aber er war dieses Mal äusserst gestresst wegen der
  Kritik. Vielleicht braucht er einfach ein wenig Ruhe.\\ Mich interessiert nur,
  wie er in dieser kurzen Zeit fertig werden konnte, und sein Entwurf auch noch
  so gut ist...}

\xdlg{Alessia}{Höre ich da etwa Eifersucht aus dem Mund des
  Musterschülers.}{neckisch}


% ---------- scene ----------------------------------------------------------- %
% ############################################################################ %
\scn{int}{WG}{Abend}{Zwischenschnitt. - TIM und ALESSIA kommen sich näher. Aus
  den neckischen Sprüchen sind bald richtige Annäherugsversuche entstanden. Nach
  dem Wegfallen der Hemmungen durch ein, zwei weitere Gin Tonic Tanzen die
  beiden nun mit einigem Körperkontakt. Alessia flüstert Tim etwas ins Ohr.}


% ---------- scene ----------------------------------------------------------- %
% ############################################################################ %
\scn{int}{ETH Mensa}{Mittag des nächsten Tages}{ALESSIA und TIM stehen in der
  Warteschlange der Mensa. JAN tritt dazu. Er wirkt immer noch müde.}

\dlg{Tim}{Na du Faulpelz. Ausgeschlafen?\\ Du warst heute morgen nicht im
  Unterricht?}

\dsc{Sie gehen zu einem freien Platz im hinteren Bereich des Speisesaals.}

\sbh{Am Tisch}

\dlg{Alessia}{Habt ihr in den Nachrichten gesehen? Die Regierung in den
  Vereinigten Staaten hat die Prohibition wieder eingeführt, um den ``Missbrauch
  von Alkohol'' Einzudämmen.}

\xdlg{Tim}{Nein, hatte ich noch nicht gehört}{glücklich lächelnd}

\dsc{Die Situation zwischen ALESSIA und TIM scheint sichtlich verändert. JAN
  bemerkt dies sofort und hackt nach.}

\dlg{Jan}{Was war noch los gestern Abend?...}

\dlg{Tim}{Na etwa das übliche. Es gab Bier, die Leute waren ausgelassen und um
  zwei sind die letzten nach hause gegangen, wie ich gehört habe. Wir sind um
  Mitternacht gegangen}

\dlg{Alessia}{Vielleicht kommst du das nächste Mal einfach wieder mit? Dann
  kannst du mit eigenen Augen sehen, was passiert. Du lässt doch sonst keine
  Sause aus.}

\dlg{Jan}{Ihr wirkt beide so entspannt heute. Was habt ihr genommen?}

\dlg{Tim}{Du weisst doch, dass wir nichts brauchen ausser uns selbst.}

\dsc{Tim sieht verstohlen zu Alessia herüber.}

\dlg{Jan}{Ihr verheimlicht mir doch etwas?!}

\xdlg{Alessia}{Warum sollten wir das tun. Dein Pech, dass du nicht da
  warst.}{grinsend}

\dsc{JAN verlässt aufgebracht die Mensa.}


% ---------- scene ----------------------------------------------------------- %
% ############################################################################ %
\scn{int}{Jans Zimmer}{Abend}{JAN ist wieder zuhause in seinem Zimmer und
  Startet seinen Computer. MIRA meldet sich sofort. Jan schaltet sie aber sofort
  aus}

\dlg{Mira}{Hallo Jan, sieh dir die neuen Entwürfe an, die ich für dich
  vorbereitet habe...}

\dsc{JAN schliesst das Programm. Er wischt energisch die Pläne des Projektes,
  welche er gestern präsentiert hat vom Tisch.}

\trns{dip to black}


% ---------- scene ----------------------------------------------------------- %
% ############################################################################ %
\scn{int}{Browns Labor}{Abend}{Man sieht eine dunkle Silhouette vor seinem
  Bildschirm. Es handelt sich um DR. BROWN. Auf dem Schirm sind eben jene Pläne
  von Jan und Mira zu erkennen, die Jan zuvor von seinem Tisch gewischt hat. Mit
  einem zufriedenen Lächeln und nicken massiert er weiter die Tasten. Sein
  Gesicht ist nicht zu erkennen.}

\fadeout

\end{flushleft}
\end{document}
