\documentclass[a4paper,ngerman,11pt]{scrartcl}

  \usepackage[utf8]{inputenc} \usepackage[T1]{fontenc} \usepackage{ngerman}

  \title{Episode I}
  \subtitle{Genesis}
  \author{Sahra Roman \& Christian Sangvik}
  \date{\today}

  \renewcommand{\familydefault}{\ttdefault} \renewcommand{\ttdefault}{pcr}

  \newcommand{\linia}{\rule{.8\textwidth}{.1pt}}

  \makeatletter \renewcommand{\maketitle}{ \texttt{
  \begin{center}
        \uppercase{\tiny{ETH Zürich
          \enspace{}|\enspace{}D-ARCH
          \enspace{}|\enspace{}Professur Arno Brandlhuber
          \enspace{}|\enspace{}Herbstsemester 2017}}\\ \linia{}\\
          \vspace*{7cm} \uppercase{\Huge{De Architectura}}\\
          \vspace*{2cm} \uppercase{\normalsize{von}}\\
          \vspace*{1cm} \uppercase{\LARGE{\MakeUppercase{\@author}}}\\
        \clearpage
          \vspace*{3cm} \textbf{\huge{\MakeUppercase{\@title}}}\\
          \vspace*{1cm}
          \Large{\textit{--\enspace{}\MakeUppercase{\@subtitle}\enspace{}--}}\\
          \vspace*{2cm}
      \end{center}
  }} \makeatother

  \newcommand{\desc}[1]{
    \vspace*{12pt}
    \noindent{}{#1}\\
    \vspace*{12pt} }

  \newcommand{\fadein}{
    \noindent\uppercase{Fade in:}\\ }

  \newcommand{\fadeout}{
    \noindent\uppercase{Fade out:}\\ }

  \newcommand{\trans}[1]{ \hfill\uppercase{{#1}:}\\ }

  \newcommand{\subh}[1]{
    \vspace*{12pt}
    \noindent\uppercase{{#1}\\}
    \vspace*{12pt} }

  \newcommand{\dialog}[2]{
    \begin{center}
    \uppercase{#1}\\
    \vspace*{6pt}
    \begin{minipage}{.8\textwidth}
    {#2}\\
    \end{minipage}\hfill
    \end{center}
    \vspace*{12pt} }

  \newcommand{\exdialog}[3]{
    \begin{center}
    \uppercase{#1}\\ ({#3})\\
    \vspace*{6pt}
    \begin{minipage}{.8\textwidth}
    {#2}\\
    \end{minipage}\hfill
    \end{center}
    \vspace*{12pt} }

  \newcommand{\scene}[4]{
    \vspace*{12pt}
    \noindent{\uppercase{#1}.\quad{}\uppercase{#2}\quad{}-\quad{}\uppercase{#3}}\\ \desc{#4}
  }


% ############################################################################ %
%                                 D O C U M E N T                              %
% ############################################################################ %

\begin{document}
\begin{flushleft}

\maketitle

\fadein

% ############################################################################ %
\scene{int}{Jans Zimmer}{Morgen}{JAN AEBERSOLD wacht auf und merkt, dass er
  verschlafen hat.  Er hat die Nacht an seinem Schreibtisch verbracht, wo er
  beim späten Arbeiten eingeschlafen ist. Im Stress packt er seine Sachen
  zusammen und eilt ohne Frühstück aus dem Haus.}

% ############################################################################ %
\scene{int}{Bus}{Morgen}{JAN sitzt müde im Bus und schaut aus dem Fenster. Er
  sieht die Stadt Zürich von oben im morgendlichen Nebel.}

% ############################################################################ %
\scene{ext}{ETH Hönggerberg}{Morgen}{JAN trifft auf TIM BERGMANN. Tim wartet
  bereits vor dem Haupteingang des HIL Gebäudes. Er raucht eine Zigarette und
  hält einen Becher Kaffee in der Hand. Als Jan Tim erreicht, betreten sie
  schnell das Gebäude.}

  \exdialog{Tim}{Na schau mal an, wer da kommt! Hast du's auch noch
    geschafft?}{grinsend}

  \exdialog{Jan}{Du hättest mir lieber auch einen Kaffee mitgebracht.}{murmelnd}

  \desc{TIM gibt JAN grinsend den zweiten Becher Kaffee, den er versteckt
    gehalten hat. Sie betreten das Gebäude.}

% ############################################################################ %
\scene{int}{Korridor - ETH}{Morgen}{Onetake - JAN und TIM gehen zusammen durch
  die Korridore in Richtung Koje.}

  \dialog{Jan}{Ich bin gestern am Schreibtisch eingeschlafen. Ich komme einfach
    nicht weiter mit meinem Projekt.\\Vielen Dank, dass du mir dabei hilfst.}

  \dialog{Tim}{Kein Problem. Ich habe ja sonst nichts zu tun.}

  \dialog{Jan}{Shut up, Bitch!}

  \desc{Sie erreichen die Koje. Nach einer kurzen Pause}

  \dialog{Jan}{Wie hast du den Cluster an der Langstrasse gelöst?}

  \dialog{Tim}{Du musst den Verkehrsparameter sehr schwer gewichten. Sonst hast
    du am Schluss eine Fussgängerzone.}

  \dialog{Jan}{Das wäre doch auch nicht so schlecht. Aber es löst das Problem
    der öffentlichen Anbindung nicht.}

\subh{ALESSIA kommt dazu}

  \dialog{ALlessia}{Guten Morgen ihr zwei! Naja mindestens dich, Jan, brauche
    ich wohl kaum zu fragen, ob du gut geschlafen hast. Kommst du voran?}

  \dialog{Jan}{Tim ist meine letzte Hoffnung. Ich habe schon so zu wenig
    Zeit. Wenn das jetzt nicht klappt, brauche ich morgen erst gar nicht zu
    kommen.}

  \dialog{Alessia}{Dann will ich euch nicht länger aufhalten. Viel Erfolg!}

  \desc{ALESSIA geht und setzt sich an ihren eigenen Platz, wo sie den Laptop
    aufklappt. JAN schaut ihr abwesend hinterher.}

  \exdialog{Tim}{So! Jetzt fertig gesabbert! Weiter an deinem Projekt.}{lachend}

  \trans{hard cut}

% ############################################################################ %
\scene{int}{GUDZILLAs Büro}{Vormittag}{DR. STANISLAV BROWN sitzt ungeduldig im
  Büro von ETH PRÄSIDENT LINO GUDZILLA und wartet. Gudzilla tritt ein.}

  \dialog{Gudzilla}{Dr. Brown. Vielen Dank, dass sie so kurzfristig kommen
    konnten.}

  \desc{Gudzilla setzt sich}

  \dialog{Gudzilla}{Nun, ich will ganz offen mit Ihnen sprechen. Sie wissen ja
    sicherlich, dass die Stadt Zürich die Gelder für die Forschung und Bildung
    gekürzt hat. Auch wenn wir immernoch den Löwenanteil davon haben, sind wir
    nichtsdestotrotz betroffen. Sie verstehen sicherlich, dass ich meine
    Forschungsgelder sehr sparsam einsetzen muss.}

  \dialog{Brown}{Ja, das ist mir zu Ohren gekommen. Aber was hat dies mit mir zu
    tun?}

  \desc{GUDZILLA wartet einen Moment, bevor er fortfährt}

  \dialog{Gudzilla}{Ihre Forschung hat bis jetzt sehr wenige Früchte
    getragen. Ich habe Ihnen bereits mehrfach einen Zeitaufschub gewährt um zu
    Resultaten zu kommen. Aber Sie sind Ihren Teil mit den Erfolgen bis jetzt
    schuldig geblieben. Sie werden verstehen, dass ich unter diesen Umständen
    Ihr Projekt nicht mehr länger finanzieren kann.}

  \dialog{Brown}{Aber ich bin nun endlich unmittelbar vor dem grossen
    Durchbruch!  Das Programm funktioniert und ist ganz kurz vor der Vollendung!
    Geben Sie mir doch nur noch eine Wo..}

  \desc{GUDZILLA unterbricht Brown}

  \dialog{Gudzilla}{Ich habe mich doch bezüglich den zeitlichen
    Rahmenbedingungen das letzte Mal klar und deutlich ausgedrückt, oder? Es tut
    mir leid, aber ich kann hier keine Ausnahme machen. Ich kann nichts mehr für
    sie tun.}

  \dialog{Brown}{Bin ich... Entlassen?}

  \dialog{Gudzilla}{Sie sind an ihrem Departement ja als Informatiker
    eingestellt. Daran ändert sich nichts. Aber als wissenschaftlichen
    Mitarbeiter muss ich sie leider freistellen.}

\trans{hard cut}

% ############################################################################ %
\scene{int}{Koje - ETH}{Vormittag}{JAN sitzt frustriert vor seinem Computer. Das
  Programm Dreamfetcher stürzt immer wieder ab. Er bittet TIM um Hilfe, aber der
  kann ihm auch nicht helfen. Sie wenden sich an ALESSIA, da beide wissen wie
  gut sie mit Computern umgehen kann.}

  \exdialog{Jan}{Verfluchte Dreckskacke!}{energisch, aufgebracht}

  \dialog{Tim}{Was hast du denn jetzt wieder angestellt?}

  \dialog{Jan}{Ich habe keinen Nerv mehr für diesen Scheiss!}

  \dialog{Tim}{Was ist denn los?}

  \dialog{Jan}{Das Programm stürzt jedes mal ab, wenn ich die Gewichtung der
    Parameter ändere.}

  \dialog{Tim}{Du solltest dir vielleicht endlich einen neuen Computer
    zulegen... Zeig mal her!}

  \desc{Jan gibt Tim seinen Laptop. Sie probieren beide ein wenig unbeholfen
    herum das Problem zo lösen, ohne jedoch Erfolg zu haben.}

  \dialog{Tim}{Hmm... Keine Ahnung. Fragen wir Ale, sie versteht sich besser
    darauf. Ale! Hast du kurz zeit?}

  \exdialog{Alessia}{Um was geht's denn?}{überrascht}

  \dialog{Tim}{Jan hat ein Problem mit Dreamfetcher, das ich nicht lösen
    kann. Kannst du dir das mal ansehen?}

  \desc{ALESSIA tritt zu Jans Computer und beginnt sofort zu tippen. Jan und Tim
    schauen gespannt zu.}

  \exdialog{Alessia}{So. Nach einem Neustart müsste es
    funktionieren.}{überzeugt}

  \desc{Jan startet den Computer neu und gibt sein Passwort ein.}

  \dialog{Alessia}{Ehrlich? Password.1234?}{augenrollend}

  \desc{Jan erwidert verlegen Alessias Blick. Er startet das Programm, welches
    sofort wieder abstürzt.}

  \exdialog{Alessia}{Hmm... Das verstehe ich jetzt nicht. Vielleicht gehst du
    doch lieber zum Helpdesk.}{verlegen}


% ############################################################################ %
\scene{int}{Helpdesk - ETH}{Nachmittag}{DR. BROWN ist schlecht gelaunt. Er sitzt
  am Helpdesk und erklärt gerade einem STUDENTEN, dass er ihm nicht helfen
  kann. JAN wartet ab bis er an der Reihe ist und erklärt dann sein Problem.}

  \dialog{Jan}{Hallo. Ich habe ein Problem mit `Dreamfetcher'. Das Programm
    stürzt immer ab, wenn ich versuche die Parametergewichtung zu ändern.}

  \dialog{Dr. Brown}{Hast du versucht den Computer neu zu starten?}{sichtlich
    genervt}

  \dialog{Jan}{Natürlich. Aber das hat auch nichts geholfen. Ich glaube, es
    handelt sich um einen Softwarefehler. Vielleicht ist es auch wegen meinem
    alten Computer}

  \desc{Jan zeigt seinen alten Computer. Brown scheint offenbar interessierter.}

  \dialog{Brown}{Zeig mal her. Ich werde mir das ansehen.}

  \desc{Brown tippt ein wenig in den Tasten herum.}

  \dialog{Brown}{Hm. Ich fürchte, ich kann da nichts machen. Es scheint
    tatsächlich ein grundlegenderes Problem zu sein. Ich kann es mir höchstens
    im Detail ansehen und sehen, was ich machen kann. Aber ich fürchte, du musst
    deinen Computer bis heute Abend hier lassen.}

  \dialog{Jan}{Aber das geht nicht. Ich habe morgen eine Kritik und muss noch
    viel arbeiten.}

  \dialog{Brown}{Dann kann ich dir nicht helfen. Ich verstehe deine Not. Aber
    alles was ich machen kann, ist dir anzubieten, dass ich heute länger arbeite
    und versuche deinen Laptop wieder zum laufen zu bringen.}

  \desc{Jan scheint entäuscht, dass es keine Lösung für sein Problem zu geben
    scheint. Als er an seine Abgabe denkt, haucht ein Anflug von Panik über sein
    gesicht. Da alles nichts zu nützen scheint willigt er resigniert ein den
    Laptop da zu lassen.}

  \dialog{Jan}{Also gut. Wenn sonst nichts geht, müssen wir es eben so
    versuchen}

  \dialog{Brown}{Keine Sorge. Ich werde deinen Computer bis heute abend fertig
    haben. Ich rufe dich an, wenn ich so weit bin.}

  \desc{JAN hinterlässt seine Telefonnummer auf einem Zettel und geht davon.}


% ############################################################################ %
\scene{int}{Alumni Lounge - ETH}{Abend}{JAN sitzt frustriert und müde am
  Tisch. Neben sich einen leeren Kaffeebecher. Er schaut immer wieder auf sein
  Handy und wartet gestresst auf den Anruf von BROWN. Gedankenverloren kritzelt
  er in ein Notizbuch und versucht sein Projekt von Hand zu skizzieren. Endlich
  klingelt das Telefon. Sein Computer ist abholbereit. JAN packt hastig seine
  Sachen zusammen und geht.}

  \exdialog{Brown}{Hallo Jan.\\ Ich habe deinen Computer nun fertig. Du kannst
    ihn nun abholen.}{über das Telefon}

  \dialog{Jan}{Super ich komme sofort!}

% ############################################################################ %
\scene{int}{Helpdesk - ETH}{Abend}{BROWN übergibt JAN den Laptop. JAN ist
  sichtlich erleichtert.}

  \dialog{Brown}{Mit der alten Installation war nichts mehr anzufangen. Ich habe
    dir deshalb die neueste Version installiert. Mit dieser hast du nun auch
    nicht mehr das Problem mit den Lizenzen.}

  \exdialog{Jan}{Vielen Dank! Ich wollte schon aufgeben.\\Aber unter Umständen
    muss ich das nun auch. Bis morgen bleibt nicht mehr viel Zeit, und es ist
    noch eine Menge zu tun.}{Glücklich, aber trotzdem mit \\einem Anflug von
    Resignation}

  \dialog{Brown}{Mach dir deswegen mal keine allzu grossen Gedanken. Du wirst
    feststellen, dass die neue Version um einiges mächtiger ist als die
    alte. Die Löst dir alle Probleme quasi von selbst.}

  \exdialog{Jan}{Dann werde ich sie wohl gleich mal ausprobieren und an ihre
    Grenze bringen. Schönen Abend noch, und nochmals vielen Dank!}{ungläubig}

  \exdialog{Brown}{Ich glaube nicht, dass du das schaffst. Das wünsche ich dir
    auch.}{lachend} \trans{dissolve to}

% ############################################################################ %
\scene{int}{JANs Zimmer}{Nacht}{JAN startet seinen Computer und dann auch gleich
  das Programm. Man sieht ihm an, dass er eigentlich überhaupt keine Motivation
  mehr hat und lieber gleich schlafen gehen würde. Sobald das Programm gestartet
  ist, fängt MIRA an zu sprechen.}

  \dialog{Mira}{Hallo! Ich bin Mira, Ihre neue Assistentin für den
    architektonischen Entwurf. Wie kann ich Ihnen helfen?}

  \desc{Jan nimmt die Maus in die Hand und versucht in der neuen Oberfläche
    einige Parameter einzustellen.}

  \dialog{Mira}{Sie können wie gewohnt Ihre Angaben im klassischen Interface
    einstellen. Eine effizientere Methode wäre jedoch mit dem neuen
    Sprachinterface.  Möchten Sie die Sprachsteuerung für Ihr Projekt starten?}

  \exdialog{Jan}{Ja ja\ldots}{ungläubig}

  \dialog{Mira}{Die Sprachsteuerung ist nun aktiviert.\\Wo soll Ihr neues
    Projekt liegen?}

  \desc{Jan zögert zunächst. Er ist überfordert mit der neuen, viel
    selbstständigeren Version von Dreamfetcher. Bald aber erkennt er, dass es
    doch noch eine Chance für ein gutes Projekt hat und bekommt wieder Hoffnung
    und neue Energie. MIRA ist wie eine Teampartnerin oder Sekretärin, die im
    Dialog mit Jan entwirft. Jan muss also nicht mehr am Computer sitzen und
    läuft im Zimmer rauf und runter während er Mira den Entwurf diktiert. Mira
    hinterfragt alles und bringt dadurch den Entwurf in eine bessere
    Richtung. Zusammen arbeiten sie bis spät in die Nacht.}

% ############################################################################ %
\scene{int}{Plotterraum}{Am nächsten Morgen}{JAN steht müde aber glücklich und
  erleichtert beim Plotter und sammelt seine Ausdrucke zusammen. TIM findet Jan
  und bringt ihm einen Kaffee.}

  \dialog{Tim}{Hat es also noch geklappt mit deinem Computer gestern? Brändi
    wird es sicherlich verstehen, dass du unter diesen Umständen den Entwurf
    nicht vollenden konntest.}

  \exdialog{Jan}{Ach, weisst du, ich bin heute früh fertig geworden}{grinsend}

  \exdialog{Tim}{Zeig mal her!}{überrascht}

  \desc{Tim schaut sich Jans Ausdrucke ein wenig näher an.}

  \dialog{Tim}{Diese Pläne hier sind richtig gut. Und wie ich sehe, hattest du
    sogar die Zeit, eine schöne Visualisierung anzufertigen. Wie hast du das nur
    gemacht?}

  \exdialog{Jan}{Ich habe die ganze Nacht daran gesessen. Offenbar zahlt sich
    beharrlichkeit und fleiss im letzten Moment durchaus aus}{ausweichend}

  \desc{Als Jan seine Ausdrucke zusammen und zugeschnitten hat, schaut er auf
    die Uhr.}

  \dialog{Jan}{Wenn wir vor der Kritik noch eine Zigarette rauchen wollen,
    müssen wir JETZT gehen.}

  \desc{Die beiden nehmen ihre jeweiligen Pläne unter den Arm und brechen auf.}


% ############################################################################ %
\scene{int}{Koje bei der Kritik - ETH}{Im Verlaufe des Tages}{JAN präsentiert
  sein Projekt. PROFESSOR BRÄNDI, GIOVANNI BENINI und eine weitere Architektin
  sitzen in der vordersten Reihe. Dahinter sammeln sich die Studenten. Jan
  beendet gerade seine Präsentation.}

  \exdialog{Jan}{Und so, glaube ich, kann die Lebendigkeit dieses Stadtteils
    bewahrt werden. Wenn man auf die neuen Bedürfnisse eingeht, aber die alten
    bewahrt}{sichtlich nervös}

  \dialog{Prof. Brändi}{Sie zeigen hier eine Reihe sehr interessanter Ansäzte.
    IMich würde interessieren, wie sie dabei vorgegangen sind. Auch da ich mich
    nicht erinnern kann diese Richtung in Ihrem bisherigen Prozess gelesen zu
    haben.  Verstehen sie mich nicht falsch, ich finde die Richtung nicht
    schlecht, aber wie sind sie zu Ihrem Sinneswandel gekommen?}

  \exdialog{Jan}{Ich war mit meinem bisherigen Ansatz nicht mehr zu frieden, und
    habe alles komplett von vorne aufgerollt.}{Nicht lügend, aber die Wahrheit
    verbergend}

  \dialog{Giovanni}{Wie sind Sie demnach bei Ihrem neuen Ansatz darauf gekommen,
    die Verteilung der Öffentlichkeit und der Erschliessung auf diese Weise zu
    lösen. Welche Parameter waren Ihnen persönlich so wichtig, dass Sie auf
    dieses Resultat kommen?}

  \exdialog{Jan}{Nun, ich will nicht allzu sehr ins Detail meiner Parameter
    gehen. Mir war es aber wichtig die Ausgewogenheit von Bestand und der
    integration vom Neuen zu erreichen. Sonst wäre meine Idee von Grund auf
    nicht aufgegangen.}{verlegen und suchend}

  \exdialog{Architektin}{Ja, ich weiss nicht richtig, was zu sagen.\\Mir gefällt
    es.\\Können Sie mir nochmals die Aufteilung im Erdgeschoss
    aufzeigen?}{kopfnickend}

  \desc{Brändi und die Kritiker sind überzeugt und zufrieden mit dem
    Projekt. Sie finden Jans Ansatz sehr schön und intelligent gelöst.}

% ############################################################################ %
\scene{ext}{Dachterrasse - Apéro}{Abend}{ JAN, TIM, ALESSIA und weitere Freunde
  stehen auf der Dachterrasse und feiern das Ende der Kritik mit einem oder zwei
  Gin Tonic.}

  \dialog{Alessia}{So, das hätten wir geschafft. Nun haben wir endlich einen Tag
    Pause.}

  \exdialog{Tim}{Du sagst es. Es war ein harter kampf bis vorgestern Abend, als
    ich fertig wurde.}{grinsend}

  \dialog{Jan}{Mensch bin ich froh, dass es vorüber ist. Aber es ist ja zum
    glück recht gut gelaufen.}

  \dialog{Tim}{Ja Jan. Wie hast du das geschafft? Ich meine, als ich mich
    gestern mit dir zusammengesetzt habe, war dein Projekt noch nirgends.. Und
    heute? Heute überflügelst du mich.}

  \desc{Tim boxt Jan freundschaftlich an die Schulter.}

  \dialog{Alessia}{Wollen wir nachher noch feiern gehen? Bei Jens findet heute
    noch eine kleine Feier statt.}

  \desc{JENS geht zufällig gerade an der Gruppe vorbei.}

  \dialog{Jens}{Ja stimmt. Es wäre schön, wenn ihr alle mitkommt. Wir brechen in
    einer halben Stunde auf. Betrachten wir dieses Apero als
    Vorgeschmack.\\Übrigens, gratuliere Jan. Dein Projekt war hammerhart.}

  \dialog{Jan}{Vielen Dank. Aber ich werde nicht mitgehen. Ich bin nach heute
    Nacht hundemüde und will nur noch in mein Bett und schlafen.}

  \dialog{Alessia}{Schade. Aber ich und Tim kommen gerne. oder?}

  \dialog{Tim}{Ja sicher! Eine Fete in Jens' WG lasse ich mir doch nicht
    entgehen.}

  \desc{Jan erhält ständig push-Nachrichten auf sein Mobiltelefon und ist häufig
    geistesabwesend. Er verabschiedet sich und geht alleine in Richtung Bus.}

% ############################################################################ %
\scene{int}{Jans Zimmer}{Abend}{JAN spricht mit MIRA. Sie will nicht aufhören zu
  lernen, und löchert JAN mit Fragen zu Architektur, deren Geschichte und den
  Menschen in Zürich.}

  \dialog{Mira}{Welches Design würden Sie persönlich für eine öffentliche
    Bibliothek bevorzugen?}

  \desc{Mira zeigt Jan auf dessen Monitor drei verschiedene Bilder die allesamt
    unterschiedliche Gebäudearten aufzeigen.}

  \dialog{Jan}{Das Haus rechts.}

  \dialog{Mira}{Würden die Menschen in Zürich ein solches Gebäude als Schule
    akzeptieren?}

  \desc{Mira zeigt Jan ein anderes Bild eines Gebäudes.}

% ############################################################################ %
\scene{int}{WG}{Abend}{ALESSIA und TIM sprechen über Jan}

  \dialog{Alessia}{Findest du nicht auch, dass sich Jan heute sehr merkwürdig
    verhalten hat?}

  \dialog{Tim}{Ja schon. Aber er war dieses Mal äusserst gestresst wegen der
    Kritik. Vielleicht braucht er einfach ein wenig Ruhe.\\ Mich interessiert
    nur, wie er in dieser kurzen Zeit fertig werden konnte, und sein Entwurf
    auch noch so gut ist\ldots}

  \exdialog{Alessia}{Höre ich da etwa Eifersucht aus dem Mund des
    Musterschülers.}{neckisch}

% ############################################################################ %
\scene{int}{WG}{Abend}{Zwischenschnitt. - TIM und ALESSIA kommen sich näher. Aus
  den neckischen Sprüchen sind bald richtige Annäherugsversuche entstanden. Nach
  dem Wegfallen der Hemmungen durch ein, zwei weitere Gin Tonic Tanzen die
  beiden nun mit einigem Körperkontakt. Alessia flüstert Tim etwas ins Ohr.}

% ############################################################################ %
\scene{int}{ETH Mensa}{Mittag des nächsten Tages}{ALESSIA und TIM stehen in der
  Warteschlange der Mensa. JAN tritt dazu. Er wirkt immer noch müde.}

  \dialog{Tim}{Na du Faulpelz. Ausgeschlafen?\\ Du warst heute morgen nicht im
    Unterricht?}

  \desc{Sie gehen zu einem freien Platz im hinteren Bereich des Speisesaals.}

  \subh{Am Tisch}

  \dialog{Alessia}{Habt ihr in den Nachrichten gesehen? Die Regierung in den
    Vereinigten Staaten hat die Prohibition wieder eingeführt, um den
    ``Missbrauch von Alkohol'' Einzudämmen.}

  \exdialog{Tim}{Nein, hatte ich noch nicht gehört}{glücklich lächelnd}

  \desc{Die Situation zwischen ALESSIA und TIM scheint sichtlich verändert. JAN
    bemerkt dies sofort und hackt nach.}

  \dialog{Jan}{Was war noch los gestern Abend?...}

  \dialog{Tim}{Na etwa das übliche. Es gab Bier, die Leute waren ausgelassen und
    um zwei sind die letzten nach hause gegangen, wie ich gehört habe. Wir sind
    um Mitternacht gegangen}

  \dialog{Alessia}{Vielleicht kommst du das nächste Mal einfach wieder mit? Dann
    kannst du mit eigenen Augen sehen, was passiert. Du lässt doch sonst keine
    Sause aus.}

  \dialog{Jan}{Ihr wirkt beide so entspannt heute. Was habt ihr genommen?}

  \dialog{Tim}{Du weisst doch, dass wir nichts brauchen ausser uns selbst.}

  \desc{Tim sieht verstohlen zu Alessia herüber.}

  \dialog{Jan}{Ihr verheimlicht mir doch etwas?!}

  \exdialog{Alessia}{Warum sollten wir das tun. Dein Pech, dass du nicht da
    warst.}{grinsend}

  \desc{JAN verlässt aufgebracht die Mensa.}

% ############################################################################ %
\scene{int}{Jans Zimmer}{Abend}{JAN ist wieder zuhause in seinem Zimmer und
  Startet seinen Computer. MIRA meldet sich sofort. Jan schaltet sie aber sofort
  aus}

  \dialog{Mira}{Hallo Jan, sieh dir die neuen Entwürfe an, die ich für dich
    vorbereitet habe...}

  \desc{JAN schliesst das Programm. Er wischt energisch die Pläne des Projektes,
    welche er gestern präsentiert hat vom Tisch.}

  \trans{dip to black}

% ############################################################################ %
\scene{int}{Browns Labor}{Abend}{Man sieht DR. BROWN vor seinem Bildschirm. Auf
  dem Schirm sind eben jene Pläne von Jan und Mira zu erkennen, die Jan zuvor
  von seinem Tisch gewischt hat. Mit einem zufriedenen Lächeln massiert er
  weiter die Tasten. Sein gesicht ist nur vom Monitor beleuchtet.}

\fadeout

\end{flushleft}
\end{document}

