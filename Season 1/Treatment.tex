% Created 2017-11-29 Mit 09:47
% Intended LaTeX compiler: pdflatex
\documentclass[11pt,a4paper,ngerman]{scrreprt}
\usepackage[utf8]{inputenc}
\usepackage[T1]{fontenc}
\usepackage{ngerman}
\addtokomafont{disposition}{\rmfamily}
\author{Sahra Roman \and Christian Sangvik}
\date{\today}
\title{Scenario}
\subtitle{Treatment}

\begin{document}

\maketitle

\chapter*{Generelle Beschreibung}

Die Serie beschäftigt sich mit den Welten, in welchen wir
möglicherweise leben würden, wenn gewisse andere Rahmenbedingungen gegeben
wären. In einer Season wird jeweils ein Szenario diskutiert. Die Welten
zwischen den einzelnen Seasons sind nicht die gleichen.

Das Ziel ist, ein konsistentes Szenario über eine komplette Season zu
beschreiben und ein möglicher Weg, hierin zu leben und Arbeiten. Innerhalb der
Season wird eine chronologische Geschichte erzählt. In jeder Episode nehmen
wir jedoch die Perspektive eines anderen Charakters ein.

In der ersten Season befinden wir uns in einer möglichen Welt in der
unmittelbaren Zukunft. Sie hat gegenüber unserer Welt einen Vorsprung von
ungefähr fünf Jahren.

Die Technik ist im Bereich der Automation und Robotik, wie auch der
eigenstängigen Software einen Schritt weiter gekommen.

\chapter*{Cast und Rollen}
\section*{Professor Arno Brändi}

\emph{Prof. Brändi} ist ein Urgestein des Lehrkörpers an der Architekturfakultät
der ETH Zürich.

Er ist zwar innovativ und hat es verstanden seine Architektur an die
Gegebenheiten der Zeit anzupassen, ist aber seinen alten Methoden treu
geblieben. Er erfreut sich grosser Beliebtheit bei den Studenten, da er mit
seiner Bestimmtheit, konstruktiven Kritik aber vor allem seiner
Menschlichkeit und Nahbarkeit punktet.

Auch unter seinen Kollegen geniesst er grossen Respekt, sowohl für seine
Architektur wie auch die Didaktik.

\section*{Lino Gudzilla (ETH Präsident)}

Der Präsident der Eidgenössischen Technischen Hochschule in Zürich.

Er ist ein ambintionierter und seriöser Mensch.

Sein Interesse gilt sowohl der Lehre, wo er eine sich weiterentwickelnde
Schule führen will, und somit auch viel Geld in die Forschung und gute
Dozenten steck, wie auch der Ökonomie, wo er sich den Geldgebern gegenüber
verpflichtet fühlt, und trotz seiner freigiebigen Ader für die Forschung
gewisse Projekte nicht unterstützen will und kann.

Er is der direkte Vorgesetzte von Dr. Brown, wessen Forschung er aus mangel
an Erfolg beenden will. Nach Dr. Browns Durchbruch wird dieser für ihn jedoch
zu einem Vorzeigeexemplar der Forschung am Haus.

\section*{Dr. Stanislav Brown (Programmierer)}

\emph{Dr. Brown} ist ein Programmierer an der ETH. Er ist ausgebildeter aber
gescheiterter Architekt, und hat sich in seinem Schaffen der Technologie
zugewendet. Im Rahmen seiner Arbeit hat er den Ansatz des CAD zeichnens
weiterentwickelt, und an einem Programm namens `Dreamcatcher'
mitgeschrieben. \emph{Dreamcatcher} ist ein Programm, welches das Design von
Architektur um ein vielfaches vereinfachen soll. Mit der Eingabe von diversen
relevanten Parametern produziert das Programm verschiedene Lösungen zur
architektonischen Gestaltung. Der Architekt wählt dann aus der Liste der
Resultate eines aus, welches er dann verfeinert.

Brown hat diese Automation nie vollends befriedigt. Für seine Forschung hat
er dieses Programm weiterentwickelt. Auch um sein Scheitern im Beruf
aufzuwiegen. Unter dem Namen ``Mira'' hat er eine künstliche Intelligenz
entworfen, welche sich die relevanten Parameter, und auch erheblich mehr,
zusammensucht und letztendlich selber entscheidet, welches das beste und
schönste Design ist, so dass das Produkt der fertige Entwurf ist.

Nach anfänglichen Rückschlägen und der Androhung des ETH Präsidenten
Gudzilla, seine Mittel zu streichen, war er am Schluss trotz allem
erfolgreich. Er geniesst seinen neuen Ruhm und stellt voller Stolz seine
Künstliche Intelligenz, bei jeder sich bietenden Gelegenheit vor. Dies unter
anderem auch im Rahmen einer grossen öffentlichen Vorlesung an der ETH
selber.

\section*{Mira (K.I.)}

Für mira wird nur eine Stimme gebraucht, sie manifestiert sich nicht
physisch.

\emph{Mira} ist eine künstliche Intelligenz. Sie ist ein Programm, welches von
Dr. Brown geschrieben wurde, welches den architektonischen Entwurf
automatisieren kann. Gegenüber einer gewöhnlichen K.I. meistert Mira nicht
nur logische Entscheidungen sondern auch \emph{intuitive}. Sie hat die Fähigkeiten
\emph{zu beurteilen} und die Welt, wie auch ihr eigenes `Denken' \emph{reflexiv} zu
betrachten und zu hinterfragen. Im Gegensatz zu anderen Erzählungen mit
künstlichen Intelligenzen hat Mira jedoch keine empathischen
Fähigkeiten. Ebenfalls tritt Mira auch nicht physisch in Erscheinung.

In Verbindung mit dem Benutzer tritt Mira über ein sprachliches Interface,
wie auch über den visuellen Kanal.

Mira ist keineswegs ein bösartiges Programm welches die Weltherrschaft an
sich reissen will, sondern nur pragmatisch interessiert, das ihr gegebene
Problem möglichst effizient zu lösen. Da sie für den Architekturentwurf
programmiert wurde, beschränkt sich ihr tun hierauf.

Hierin geht sie aber um ein vielfaches weiter als ein Vorgängerprogramm von
ihr namens `Dreamcatcher'. Sie sucht sich Entwurfsparameter eigenständig
zusammen und entscheidet nicht nur basierend auf logischen Daten, welcher
Entwurfsansatz weiter zu entwickeln sei. Ästhetische Vorlieben der
Gesellschaft und des Kontext fliessen bei ihrem Entwurf ebenso ein, wie die
Anforderungen des Baurechtes.

Zu beginn verhält sich \emph{Mira} wie ein Kind, welches alles lernen und erfragen
muss. Sie muss sich ihre eigene Wissensdatenbank anfertigen und vergisst
niemals. Ebenso denkt sie alle Ansätze weiter. Mit wachsendem Wissen ist sie
dann in der Lage, selber kreative und konstruktive Entscheidungen zu treffen.

Prinzipiell macht Mira die Entwicklung eines Menschen durch. Dies jedoch in
kürzester Zeit, weshalb sie vielmehr die Entwicklung \emph{aller} Menschen
durchmacht.

Da sie sich nicht linear entwickeln muss ist sie gleichsam eine einzelne
Entität, die jedoch wie ein komplettes globales Netzwerk funktioniert.

Limitierungen hat die K.I. jdeoch immernoch. Dies vor allem im künstlerischen
Aspekt. Auch steht die Frage noch offen, was denn beim \emph{Scheitern} an einem
Projekt passiert.

\section*{Alessia [Ale] Benini}

\emph{Alessia} ist eine Studentin am Lehrstuhl Brändi.

Sie kommt aus gutem Haus, hat in ihrem Leben viel Wohlstand genossen, ist
aber trozdem nicht zu einem verwöhnten Mädchen geworden. Ihre Eltern sind
relativ streng in der Erziehung, haben ihr nichts in den Schoss gelegt, und
sie musste sich immer einsetzen, um ihre Ziele zu erreichen. Deshalb ist sie
selbstbewusst, mutig und lässt sich nicht schnell unterkriegen. Sie braucht
eigentlich nur sich selbst um über die Runden zu kommen.

Sie ist tüchtig und erfolgreich, sowohl im Sozialen, wie auch im Studium.

Im Studium hat sie sich mit ihren Kommilitonen Jan und Tim
angefreundet. Obwohl die drei grund verschieden sind haben sie trotzdem eine
gemeinsame Basis für ein gutes zusammensein gefunden.

\section*{Tim Bergmann}

\emph{Tim} ist ebenfalls Student am Lehrstuhl Brändi.

Er ist der Musterschüler jeder Klasse. Er ist intelligent und versteht
Zusammenhänge häufig schneller als jeder sonst. Da er sehr hilfsbereit und
empathisch ist, ist er äusserst beliebt bei den anderen Studenten.

Mit seinem Engagement in der Hochschulpolitik trägt er zum Wohle aller bei.

Jan ist seit langer Zeit Tims bester Freund.

\section*{Jan Aebersold}

\emph{Jan} war in seinem Leben nicht immer gut gestellt. Er lebt zusammen mit
seiner alleine erziehenden Mutter in einer kleinen Wohnung.

Er ist sympathisch und zugänglich. Sein Fokus in seinem Leben liegt in seinem
sozialen Umfeld. Architektur ist für Jan nicht nebensächlich, er ist aber
nicht besonders gut im Studium. Widerum ist er auch nirgends wirklich
schlecht. Wenn es um die schulischen Leistungen geht, ist er die Inkarnation
von \emph{durchschnittlich}.

Er macht sich Probleme, wo keine sind, und vermag es nicht allzu gut sich auf
das wesentliche zu konzentrieren und leidet häufig unter seinem schlechten
Zeitmanagement.

\section*{Studenten}

Das Gros der Studenten. Wir fokussieren hier auf die Studenten des
Lehrstuhles Brändi. Es werden daher ca. 10 bis 20 Einzelne Studenten
benötigt.

\subsubsection{Dreigespann}

Die drei Studenten Alessia, Tim und Jan stehen in einer Art
Dreiecksbeziehung, wo Spannungen auf verschiedenen Ebenen bestehen.

Die drei Protagonisten hier sind in unserer Geschichte für die
zwischenmenschliche Ebene zuständig. Eine komplexe Liebesgeschichte wird
angedeutet.

Die drei könnten grossen Enfluss auf weitere Gestaltung der
Architekturausbildung haben.

\section*{Stadtpräsidentin Corinne Schmauch}

\emph{Schmauch} ist eine sehr zielstrebige Person. Sie erreicht ihre Ziele
eigentlich immer. Politisch aktiv ist sie seit ihrer eigenen Zeit an der
Mittelschule.

In ihrem Privatleben ist sie aber eine sehr herzliche Person und führt mit
ihrem Mann eine glückliche Beziehung.

Aktuell muss sie für ihre Wiederwahl kämpfen, und setzt Mira als
Wahlkampfmittel ein, da Mira gut ankommt bei der Bevölkerung.  Übergibt Amt
des Städtebaus an Mira. Oder reisst Mira es an sich?

\section*{Giovanni Benini (Vater von Ale)}

\emph{Giovanni} ist der Vater von Alessia. Er ist seit langer Zeit glücklich
verheiratet und wohnt zusammen mit seiner Frau und seinen zwei Kindern,
Alessia und ihr jüngerer Bruder, in einem grossen Haus in einem gehobenen
Gebiet der Stadt.

Während der Mira-Krise verliert er jedoch seinen Job. Er möchte Alessia dazu
bewegen, ihr Studium abzubrechen, obwohl er weiss, dass dies ihr Traumberuf
ist, da es in der Architektur keine Zukunft zu geben scheint.

Vor der Krise jedoch ist er selber passionierter Architekt und kandidiert für
das Amt des Direktors des Amtes für Städtebau. Um zum Amt zu kommen, neigt er
in der Phase vor der Krise dazu, viel Zeit im Büro zu verbringen.

Er ist ein wenig strikt und formalistisch und überaus ambitioniert.  Er ist
zwar herzlich, aber hat Probleme, Gefühle zu zeigen.

Privat vermag er es die Arbeit sehr gut vom Leben mit seiner Famile
abzutrennen.

Neben Alessia haben er und seine Frau noch einen jüngeren Sohn. Alessia ist
aber das Vorzeigekind. Der jüngere Sohn Luca rebelliert zuhause und
interessiert sich nicht für Architektur.

\section*{Architekten}

Eine kleine Gruppe von Architekten.

\section*{Medien Zürich}

Einige Journalisten, die bei Pressekonferenzen dabei sind und ein
Fernsehteam.

\section*{Zürcher Bevölkerung}

Eine Gruppe Zürcher Stadtbewohner

\chapter*{Season 1 | Mira}

Die erste Season wird in acht Episoden erzählt. Jede aus der Sicht eines
anderen Protagonisten. Die Hautpfigur der ersten Season ist jedoch zweifellos
Mira, die künstliche Intelligenz.

Es geht um die Geschichte der Architekten, Architekturstudenten und die Rolle
der Technik in der Gesellschaft.

Die Geschichte spielt in der nahen Zukunft, circa fünf Jahre von uns
entfernt. Die Gegeben- und Gepflogenheiten in der Gesellschaft sind den
unseren weitestgehend ähnlich, nur hat sich das Handwerk der Architekten
einigermassen geändert.

Die Architekten und Architekturstudenten brauchen nicht mehr den ganzen
Entwurf von Hand zu machen, oder zumindest nicht mehr von Hand
einzugeben. Mit einem Programm namens \emph{Dreamcatcher} ist es möglich,
Parameter eines Projektes zu beschreiben, anhand welcher der Computer
selbstständig Designs erarbeitet. Diese werden dann von den Architekten
eingesehen und beurteilt. Vielversprechende Ansätze werden dann manuell
weiterentwickelt.

Das Studium der Architektur ist aber zum Zeitpunkt der Geschichte prinzipiell
immer noch das selbe, welches wir gewohnt sind. Der Hauptunterschied liegt
lediglich darin, dass wir weniger Zeit darauf verwenden, die Gedanken in
Pläne zu übersetzen, da dieser Prozess mittels Software weitgehend
automatisiert wurde.

Forschung im Bereich der Künstlichen Intelligenz und Softwareautomation
werden an der ETH Zürich gross geschrieben.

Ein Entwickler an der ETH, \emph{Dr. Brown}, der seines Zeichens auch
ausgebildeter Architekt ist, es jedoch nie richtig geschafft hat in der Welt
der Architekten Fuss zu fassen, hat sich der Automation des Entwurfsprozesses
verschrieben. Er hat bereits an Dreamcatcher mitgeschrieben, und ist in
seinem Forschungsprojekt nun damit beschäftigt, die Software grundlegend
weiter zu entwickeln und sie mit den Ansätzen der Künstlichen Intelligenz zu
paaren. So dass am Schluss der Computer nicht eine Auswahlsendung an
verschiedenen Entwurfsgrundlagen basierend auf der logischen Interpretation
relevanter Parameter entsteht, sondern aus komplett eigenem Schaffen des
Computers der fertige Entwurf resultieren soll. Unter dem Codenamen \emph{Mira}
hat er also eine Künstliche Intelligenz für die Architektur geschrieben.

\emph{Miras} Handlungsfeld ist ausschliesslich an die Architektur gebunden. Sie
soll keine Künstliche intelligenz werden, welche allgemeine Probleme lösen
soll, diejenigen der Architektur aber im Detail.

\emph{Mira} wird, nachdem der Präsident der ETH, \emph{Gudzilla}, die Mittel der nicht
von grossen Erfolgen gekürten Forschung von \emph{Dr. Brown} streichen will, aus
\emph{Dr. Browns} Labor gestohlen. Interne Ermittlungen wegen dieses Diebstahles
werden eingeleitet, versiegen jedoch bald im Nichts.

In einem öffentlichen Architekturwettbewerb der Stadt Zürich wird später ein
Beitrag abgegeben, der die anderen um ein vielfaches überflügelt und
gewinnt. Es stellt sich heraus, dass dies der Beitrag von \emph{Mira} ist. Eine
Grundsatzdebatte über das Paradigma einer künstlichen Intelligenz an einem
Wettbewerb und deren Zulassung wird angebrochen.

Die Jury der Stadt, unter der Leitung von \emph{Giovanni}, der Anwärter auf das
frei werdende Amt des Direktors für Städtebau der Stadt Zürich ist, ringt
sich unter Skepsis und Begeisterung dazu durch, das Projekt zu zu lassen, und
die Künstliche Intelligenz mit der weiteren Ausführung zu beauftragen.

Als der Erfolg der K.I. publik wird, wird auch deren erschaffer, \emph{Dr. Brown}
von den Medien heimgesucht. Er geniesst seine neu erlangte Berühmtheit und
stellt sein Werk gerne und umfassend vor.

Nach diesem Durchbruch stellt sich \emph{Gudzilla} vollumfänglich hinter \emph{Brown}
und verwendet diesen als Vorzeigebeispiel der Forschung an der ETH.

Die Zürcher sind der Neuerung zum grössten Teil extrem positiv
gegenüber. Durch \emph{Mira} und ihre effizienten Ansätze können die Kosten für
Planung und Erstellung eines Gebäudes extrem gesenkt werden.

Gleichzeitig wehren sich aber bereits einige Architekten gegen die Neuerung,
da sie das Gefühl haben, sie könnten durch eine künstliche Intelligenz
obsolet werden.

Dies geschieht auch einigermassen. Da Mira mit der Ausarbeitung von
Ausführungsplänen viele Schritte eines Architekten selbstständig erledigen
kann.

Im Rahmen der weiteren Rationalisierung übernimmt \emph{Mira} letztendlich in
geheimer Zustimmung von Stadtpräsidentin \emph{Schmauch} das gesamte Amt für
Städtebau der Stadt Zürich.

Mittlerweile läuft \emph{Mira} auf vielen unterschiedlichen Computern, die
untereinander vernetzt sind. So lernt \emph{Mira} äusserst schnell und wird immer
noch besser und effizienter als Architekt. Das Verteilt-sein auf vielen
Computern macht zudem ein eigentliches schliessen des Programmes quasi
unmöglich.

Mit der Zeit hat \emph{Mira} sich viele Feinde gemacht, da durch sie viele
Menschen ihre Beschäftigung verloren haben. Es gibt Anschläge auf sie, welche
aber allesamt erfolglos bleiben. Am prominentesten dabei sind die grossen
Studentenaufstände, die letztendlich das Ziel verfolgen, sich eine eigene
Zukunft zu geben.

Das Gros der Bevölkerung ist aber immernoch begeistert von den Möglichkeiten,
die Mira bietet, da so viel Geld anderweitig benutzt werden kann, was sonst
nicht möglich wäre.

Letztendlich scheitert \emph{Mira} aber an ihren eigenen Ansätzen. Durch den
Versuch, das Bauen so sehr zu beschleunigen, und die Möglichkeit alles
anstehende quasi zeitgleich abzuarbeiten, scheitert Mira an der
Infrastruktur, die nicht im nötigen Mass gewachsen ist um eine ganze Stadt
gleichzeitig umzubauen. Der Verkehr kommt zum erliegen und in der Stadt
bricht ein kleines Chaos aus.

Die Studenten schaffen es mit der Hilfe von Dr. Brown Miras Möglichkeiten
einzudämmen und sie im Rahmen zu halten.

Unter Prof. Arno Brändi wird das Studium grundlegend neu strukturiert. Die
Menschen müssen lernen mit künstlichen Intelligenzen umzugehen, da sicherlich
neue erscheinen werden. Die Architekten müssen nur herausfinden, in welchen
Bereichen sie der Maschine überlegen sind, und wo sie folglich nicht
überflüssig gemacht werden können. Gleichzeitig sollen sie aber auch profit
aus den Möglichkeiten mit dem Umgang mit künstlichen Intelligenzen ziehen.

Brändi vermittelt so zwischen alt und neu in eine Richtung die nachhaltig
ist.

Als Brändi stirbt, wird diese Entwicklung aber beibehalten und die Zukunft
kann anbrechen.

Paralell dazu entwickelt Dr. Brown bereits an einer Weiterentwicklung von
Mira. Mira 2.0 wird möglicherweise bald Realität.

\chapter*{Episoden}

\section*{Episode 1 | Genesis}

Die erste Episode wird aus der Perspektive von \emph{Jan Aebersold} erzählt.

Jan wacht eines dienstagmorgens an seinem Schreibtisch auf. Er hat versucht
die Nacht durch zu arbeiten, ist dabei aber eingeschlafen. Der Grund für
seinen Eifer ist die kommende Kritik am Mittwoch Vormittag.

Jan ist mit seinem Projekt noch lange nicht so weit, dass er etwas zu
präsentieren oder besprechen hätte. Er schafft es einfach nicht die für
dieses Projekt notwendigen Parameter richtig einzustellen, so dass sich ihm
ein stimmiges Resultat offenbaren würde.

Daher hat Jan sich mit seinem besten Freund Tim verabredet. Tim soll Jan
helfen einen Ansatz zu finden, damit dieser seinen Entwurf weiterentwickeln
kann. Die Zeit dafür hat Tim, da er seinen eigenen Entwurf immer schon Tage
vor der Abgabe fertig hat. Er ist von seiner Arbeitsmoral her das pure
Gegenteil von Jan.

Hastig wirft Jan alle Sachen, die er für den Tag braucht in seinen Rucksack
und macht sich auf den Weg an die ETH. Da er für seine Verabredung mit Tim
späht dran ist, warted dieser bereits auf Jan.

In der Koje versuchen die beiden gemeinsam für Jan einen Ansatz zu
generieren, den er dann weiter verarbeiten kann. Leider kann sich Jan in der
Anwesenheit von Alessia, einer Komilitonin sehr leicht ablenken.

Parallel dazu sehen wir die Geschichte von Dr. Brown. Brown ist
Softwareentwickler an der ETH und hat im Rahmen seiner Forschung eine
Künstliche Intelligenz entwickelt, welche jedoch noch nicht ganz fertig
ist. An diesem Morgen hat Brown ein Treffen mit dem Präsidenten der ETH, Lino
Gudzilla. Gudzilla erklärt Brown, dass er seine Forschung aus Knappheit an
Forschungsgeldern und mangels Erfolgen von Brown nicht mehr finanzieren wird,
und stellt Brown als wissenschaftlichen Mitarbeiter frei. So bleibt Brown nur
noch seine Stelle an der ETH, wo er als Helpdeskmitarbeiter für
Computerprobleme den Studenten mit ihren technischen Schwierigkeiten zur
Setie steht.

Alle Versuche Gudzilla zu überreden, ihm einen Aufschub zu gewähren schlagen
fehl.

Unterdessen muss sich Jan zu allem Überfluss noch mit eben solchen
technischen Schwierigkeiten herumschlagen. Sein Parameterdesign-Programm
`Dreamfetcher' stürzt ständig ab. Auch Tim und Alessia, die sehr gut mit
Computern umgehen kann, können ihm nicht helfen, weshalb er sich gezwungen
fühlt, den Helpdesk aufzusuchen.

Brown am Helpdesk sieht im alten Computer Jans die perfekte Gelegenheit seine
noch nicht fertige K.I. auszuprobieren, um letztendlich mit offensichtlichen
Erfolgen trotzdem wieder als wissenschaftlicher Mitarbeiter eingestellt zu
werden. Er erzählt Jan also, dass er das Problem bis zum Abend beheben
werde. Jan kommt in eine riesige Not, da er so seine Abgabe niemals schaffen
wird. Resigniert stimmt er aber dennoch zu, da dies die letzte Chance auf
Erfolg ist.

Brown installiert die K.I. namens `Mira' auf Jans Computer, und meldet sich
bei ihm, dass er seinen Computer abhohlen kann. Er macht Jan glauben, er habe
lediglich eine neuere Version von Dreamfetcher installiert, die jedoch viel
mächtiger sei.

Jan probiert zuhause noch das schlimmste zu vermeiden, und ist überrascht,
wie eigenständig das Programm funktioniert. Mittels Sprachsteuerung ung der
Eigeninitiative der K.I. gelingt letztendlich der Vollständige Entwurf seiner
Abgabe. Noch dazu ist sie in diesem Fall nicht wie sonst besonders
durchschnittlich sondern überragend.

Seine Kritik läuft äusserst gut, und alle sind überrascht. In der Jury sitzen
neben Prof. Brändi noch Giovanni Benini vom Amt für Städtebau und eine andere
etablierte Architektin. Abends als die anderen Studenten ihren kleinen Erfolg
begiessen wollen, meldet sich Jan, der sonst für solche Dinge stets an
vorderster Front steht ab. Mira verlangt in ihrer Lernphase viel
Aufmerksamkeit und beansprucht so viel von Jans Zeit.

An diesem Abend kommen sich Tim und Alessia näher. Jan fällt am nächsten Tag
sofort auf, dass etwas anders ist. Jan und Tim haben eine Auseinandersetzung,
wo es um die Eifersucht gegenüber des jeweils anderen geht.

Ohne auf eine richtig gute Lösung gekommen zu sein gehen die beiden
auseinander. Zuhause versucht Mira wieder von Jans Wissen zu profitieren. Er
ist aber nicht in der Stimmung und klappt den Laptop zu.

Auflösend sieht man am Schluss Brown hinter seinem Monitor sitzen, wo die
Pläne angezeigt werden, welche Jan tags zuvor präsentiert hat.

\section*{Episode 2 | Giovanni}

Die zweite Episode wird aus der Perspektive von \emph{Giovanni Benini} erzählt.

Man sieht Giovanni zuhause. Seine Tochter Alessia, sein Sohn Luca und seine
Frau Laura leben alle gemeinsam im Hause. Die Verhältnisse zu Hause sind
grösstenteils harmonisch. Nur zwischen Alessia und Luca gibt es hin und
wieder Rankereien und Rivalitäten. Dies, weil die elterliche Erziehung streng
ist, und von beiden Leistungen erwartet werden. Giovanni hält die Ausbildung
für etwas des wichtigsten des Lebens.

Da Alessia ein Studium in Angriff genommen hat, und dort auch immer gute
Leistungen erzielt, wird sie oft als Vorbild für Luca vorgehalten, was
alleine schon diese Rivalität mitbeeinflusst.

Nach der morgendlichen Routine begibt sich Giovanni zur Arbeit. Am
Arbeitsplatz spürt man auch die freundliche Art unter den Mitarbeitern, denn
Giovanni hält nicht viel davon unmenschlich zu sein. Allerdings schwingt auch
immer Respekt und eine stilvolle, untergiebige Art im Umgang seiner Kollegen
zu ihm mit. Er nimmt seine Pflichten als Abteilungsleiter ernst, und kümmert
sich stets speditiv und rasch um alles was ansteht, denn er aspiriert für das
frei werdende Amt des Direktors des Stadtbauamtes in Zürich. Diesbezüglich
werden ihm gute Chancen beigemessen.

Aktuell soll die Jurierung des erst jüngst abgehaltenen anonymen Wettbewerbes
vorbereitet werden. Man sieht die Jurymitglieder und andere Kollegen des Amts
für Städtebau gemeinsam über die diversen Einreichungen diskutieren.

Im Verlaufe der Jurierung stellt sich ein Projekt immer mehr in den
Vordergrund. Dieses Projekt ist herausragend, und erfüllt als einziges im
ganzen Teilnehmerfeld alle Bedingungen. Ausserdem spricht die geforderte
Abschätzung der Kosten für den Bau des Projektes eine ganz andere Sprache als
die anderen Beiträge. Nur gut die hälfte der Baukosten des zweitgünstigsten
soll das Projekt kosten. Dies macht die Jury natürlich vorerst skeptisch,
aber nach mehrmaligem überprüfen scheinen die Zahlen plausibel.

Die Jury kürt folglich logisch das Projekt zum Sieger der Auslobung. Als
Giovanni nun nachsieht von wem der Beitrag stammt, staunt er nicht schlecht,
dass er über das Büro `Mira' noch nie etwas gehört hat. Nach kurzen
nachforschungen kommt Giovanni aber auf den richtigen Autor. Der Beitrag
wurde von einer Maschine eingereicht.

Als dies bekannt wird, werden alle Schritte eingeleitet, den Wettbewerbssieg
zu widerrufen.

Bei einer ausserordentlichen Sitzung beraten sich die Architekten, wie nun zu
verfahren sei. Es entbrandet eine Grundsatzdiskussion über die Maschine und
deren Rolle bei Wettbewerben und im Gewerbe generell. Sollen künftig beiträge
von Programmen berücksichtigt werden?

In der Diskussion gibt es viel dafür und dawider. Gute Argumente aus beiden
Lagern werden angeführt. Letztendlich ringen sich die Architekten unter dem
Urteil von Giovanni durch, dem ganzen einen Versuch zu gestatten. Mira soll
unter Beweis stellen, wie sie ihre versprochen tiefen Kosten einhalten kann,
und soll den Wettbewerb für die Ausführung ausarbeiten.

\section*{Episode 3 | Dr. Brown}

Die dritte Episode wird aus der Perspektive von \emph{Dr. Stanislav Brown}
erzählt.

Zu Beginn sieht man Dr. Brown, wie er die Fortschritte von Mira, und damit
auch Jan überwacht. Brown scheint zufrieden mit den Fortschritten, die sein
Programm während der letzten Stunden gemacht hat. Sein ausgeklügeltes
Lernmodul scheint gut zu funktionieren, und auf seine
Entscheidungsalgorithmen ist er stolz.

In den Medien ist ein plötzliches, riesiges Interesse an der künstlichen
Intelligenz erwacht. Ab dem Zeitpunkt wo klar wurde, dass eine K.I. einen
Architekturwettbewerb gewonnen hat wollten alle über die Sensation
berichten. Die Umstände, dass die K.I. keinen Autor hat, der sich zu ihr
bekennt macht die ganze Geschichte noch spannender und sichert Quoten in den
Nachrichten wie zu Prime-Time-Zeiten.

Alle Spuren deuten Darauf hin, dass die K.I. aus einem Labor der ETH
stammt. Es wird offenkundig, dass das Programm \emph{Mira} aus einem Labor der
Robotik und Informatik des D-ARCH stammt, wo es scheinbar zuvor entwendet
wurde. Sicherheitsdebatten kommen auf, aber nichts vermag die Sensation zu
überbieten, welche die K.I. vollbracht hat.

Mit steigendem Stolz gibt sich Dr. Brown nach einiger Zeit endlich als Autor
von Mira zu erkennen, verurteilt öffentlich den Diebstahl, hebt aber vor
allem die Errungenschaften und Vorzüge von Mira hervor. Die Berichterstattung
geht um die Welt und sorgt überal für Sensation. Natürlich gibt es immer
schon zu Beginn von etwas neuem Skeptiker, aber die Grundstimmung ist doch
sehr euphorisch.

Brown wird vielerorts eingeladen Mira vorzustellen und gemeinsam mit
prominenten und weniger prominenten zu diskutieren. Sei dies im Fernsehen
oder auch an Vorträgen und Schulen. Die ETH kann in diesem Trend natürlich
nicht hinten anstehen und veranstaltet eine Podiumsdiskussion.

Unter aller positiver Reaktion kann man hier im Hase aber schon eine grössere
Dichte an skeptischer Stimmen erkennen. Sie sind mira nicht generell negativ
entgegengestellt, hinterfragen sie jedoch mehr, als sie nur auf einen Sockel
der Errungenschaft zu stellen. Einige Architekturstudenten, darunter auch Tim
stellen ungemütliche Fragen, so dass Brown am Ende froh ist, dass die
Veranstaltung vorüber ist.

Unterdessen erfährt Gudzilla im Rahmen der internen Ermittlungen zum
Diebstahl von Mira aus dem Forschungsumfeld, dass Brown sie gestohlen hat. Er
möchte ihn aus taktischen Gründen nicht jetzt schon blossstellen, da der
Rummel viel positives Momentum in die Forschungskassen der ETH gebracht hat,
welches er nicht verspielen will. Ausserdem kann die ETH noch etwas mehr
positive Engramme in den Köpfen der Menschen brauchen. So behält Gudzilla
diese Erkenntnis vorerst für sich.

Brown wird auch an das MIT eingeladen, und bekommt dort auch schon im Voraus
ein angebot für die Forschung. Die Amerikaner, die der Entwicklung wesentlich
weniger skeptisch gegenüberstehen, als die Europäer, bejubeln Brown im
grossen Stil. Am Ende seiner Referatreihe kommen Vertreter von riesigen,
äusserst reichen Konzernen der digitalen Privatwirtschaft auf Brown zu, und
versuchen sich gegenseitig auszustechen und ihn für ihr jeweils eigenes
Unternehmen zur Weiterentwicklung von Mira zu gewinnen.

Als Brown vor hat der ETH nun den Rücken zu kehren und zu kündigen, um eines
der vielen Angebote anzunehmen, wird er von Gudzilla aber erpresst und zum
bleiben gezwungen. Er kann es sich schliesslich nicht leisten, dass sein
Diebstahl publik wird. Er wird zu einem etwas gekürzten gehalt wieder als
wissenschaftlicher Mitarbeiter eingestellt.

\section*{Episode 4 | Stadtpräsidentin Schmauch}

Die vierte Episode wird aus der Perspektive der Zürcher \emph{Stadtpräsidentin
Corinne Schmauch} erzählt.

Man sieht, wie die tüchtige Präsidentin Schmauch aus dem geschäftigen Alltag
mit vielen Telefonaten und Terminen nach Hause kommt. Mit dem übertreten der
Türschwelle wird sie gleichsam ein anderer Mensch. Im Privatleben mit ihrem
Mann zeigt sie eine unglaublich Menschliche Seite, die mit ihrem harten
Auftreten im Geschäftsalltag nichts gemeinsam hat. Liebevoll essen die beiden
und verbringen einen schönen, entspannten Abend.

Am nächsten Morgen steht schon wieder Wahlkampf an. Schmauch will im Amt
bleiben, und muss sich so die Gunst der Bevölkerung ständig neu
verdienen. Die Abstimmung über die Überbauung war im Vorfeld als Routine
eingeplant gewesen. Da nun aber ein riesiger Rummel um das Siegerprojekt und
den Umstand, dass dieses nicht aus der Hand eines Architekten oder Büros
stammt sondern aus dem Hauptspeicher eines Programmes mit künstlicher
Intelligenz ist von beiläufiger Routinehandlung nichts zu spüren. Schmauch
muss eben in solchen Situationen mit feinem Fingerspitzengefühl punkten, wenn
sie ihr Amt auch in Zukunft innehaben will.

Zu ihrer Überraschung scheint die Reaktion auf das Projekt durchwegs
positiv. Die Menschen der Stadt scheinen begeistert von der Effizienz und den
Möglichkeiten kosten einzusparen. So kann mit dem gleichen Budget viel mehr
erreicht werden. Schmauch, die diese Stimmung sehr schnell wahrnimmt, will
sich dieses Momentum zu Nutzen machen, und schwimmt mit der Welle der
Euphorie mit.

So gestärkt gewinnt sie die Wiederwahl mit für Wahlverhältnisse beachtlichem
Vorsprung. Es wird klar, dass sie bereits in der Vergangenheit vieles richtig
gemacht hat, sie sich aber durchaus versteht aus aktuellem Kapital zu
schlagen.

Nach einer Feier für ihre Wiederwahl im kleinen Kreise ihrer Freunde und
Familie wird sie am nächsten Tag aber wieder gefordert. Der Stellvertretende
Direktor des Amtes für Städtebau sucht sie ausserordentlich zu einem
dringlichen Gespräch auf. Giovanni Benini beklagt sich bei ihr, dass den
Mitarbeitern im Stadtbauamt die Hände gebunden sind, da sie kaum etwas machen
können und auf wichtige Pläne und die Serverstruktur nicht zugreiffen
können. Mira hat offenbar grosse Teile der Administration in ihren eigenen
Bereich übertragen und regelt diese nun eigenständig. Auch überbringt
Giovanni die Mitteilung, dass sich viele Architekten der Stadt bei ihm
darüber beschwert haben, dass sie kaum zu neuen Aufträgen kommen und sogar
bereits bestehende Aufträge abgezogen werden aus Gründen der
Finanzoptimierung der Bauherren.

Schmauch gesteht ein, dass sie zu wenig im Bild ist, sie ist aber gewillt der
Sache auf den Grund zu gehen und nimmt Kontakt mit Mira auf. In ihrer
gemeinsamen Unterhaltung zeigt Mira der Präsidentin auf, wo sie bisher
Optimierungen vorgenommen hat, und legt eindrücklich dar, wie viel Gelder sie
so bereits einsparen konnte, ohne jemals auf Qualität zu verzichten. Im
Gegenteil, ihre Projekte scheinen durchdachter und ergiebiger zu sein für die
Benutzung der Menschen und punkten mit passenden formalen Ansätzen für das
jeweilige Quartier, wo sie gedacht sind. Es fällt Schmauch schwer, von all
diesen Vorteilen abzulassen, und so gewährt sie Mira ihr Handeln
fortzusetzen.

Eine Welle der Empörung bricht über Schmauch zusammen, als öffentlich wird,
dass es im Amt für Städtebau Massenentlassungen geben soll. Die Posten die
nicht unbedingt gebraucht würden, sollen gestrichen werden. So zeigt sich
nach und nach, dass Mira die Kontrolle über das Amt für Städtebau nun
vollständig an sich gerissen hat.

\section*{Episode 5 | Alessia}

Die fünfte Episode wird aus der Perspektive von \emph{Alessia Benini} erzählt.

Zu Beginn sieht man, wie Alessia Feuer und Flamme für ihre Rolle als
angehende Architektin ist. Sie ist im Studium äusserst engagiert und auch bei
allen Komillitonen beliebt. Sie scheut sich nicht auch mal für das Wohle
aller mehr zu machen, sondern gieniesst insgeheim jeden Moment, in dem sie
ihren grossen Traum vom Architekt-Sein ausleben kann. Ihr Stundenplan ist so
voll wie keiner der anderen. Nach einem intensiven Tag geht sie erfüllt nach
Hause.

Zu Hause aber hängt der Haussegen schief. Giovanni ist sehr aufgebracht und
wütend. Zudem mischt sich eine grosse Verzweiflung in das Gefühlschaos,
welches man klar wahrnehmen kann. Giovanni hat im Rahmen der Rationalisierung
des Amtes für Städtebau seine Anstellung verloren. Dies kommt besonders
überraschend, da ihn insgeheim alle schon als den nächsten Direktor für das
Amt gesehen haben.

Am schlimmsten für Giovanni ist es jedoch, dass er das Gefühl hat, er müsse
sich selbst die Schuld für die jetzige Situation geben, da er ja massgeblich
daran beteiligt war, dass die Pläne der künstlichen Intelligenz am Wettbewerb
überhaupt zugelassen wurden. Nun scheint für ihn alles so auswegslos. Seine
Welt droht auseinander zu brechen, und wird nur durch das starke Netz der
Familie gehalten, auch wenn diese Situation für alle eine immense Belastung
darstellt.

Giovanni sieht offen gestanden keine Zukunft mehr für irgendjemanden in der
Architektur, da das Feld scheinbar innerhalb kürzester Zeit an die Maschine
gefallen ist. Er spricht mit einer Energie mit Alessia, die sie von ihm
überhaupt nicht kennt, und fordert sie auf, ihr Studium zu wechseln.

Mit Luca scheint Giovanni unfairer weise versöhnlicher umzugehen. Dieser
musste sich immer anhöhren was für einen exzellenten Weg seine Schwester
eingeschlagen hatte, wo er nie hatte mithalten können. Doch unter der
veränderten Situation scheint der Handwerkliche Beruf letzten Endes doch die
bessere Wahl gewesen zu sein.

Alessia kommt in eine innere Krise. Sie möchte sich sicherlich nicht gegen
ihren Vater stellen, doch kommmt für sie auch nicht in Frage, ihren
beruflichen Lebenstraum einfach so aufzugeben. In ihrem inneren Konflikt, der
immer noch belastender zu werden scheint grenzt sie sich immer mehr von ihren
Freunden ab.

Die Wendung kommt für sie von einer sehr unerwarteten Seite. Es ist plötzlich
Luca der mit einer versöhnlichen Art ankommt. Er versteht ihre Not, und
möchte sie unterstützen, auch wenn er konkret nicht genau weiss, wie das
aussehen soll. Für Alessia ist dies zumindest eine Aufmunterung in sich und
sie schöpft neue Kraft. Sie will nicht kampflos aufgeben.

Alessia beginnt zu rebellieren. Im Unterricht, den sie weiterhin besucht,
versucht sie nicht mehr integrative Wege zu fahren, sondern harte,
Konfrontationsorientierte Spuren einzuschlagen.

Tim scheint sichtlich verstört von Alessias neuer Art. Nach kurzer Zeit
vertraut sie sich ihm an. Sie erzählt ihm vieles von ihrer Not, der Situation
zu Hause, und ihren Ängsten, wenn sie in die Zukunft blickt. Sie erzählt ihm
überdies auch Details über die Umstände in der Regierung, Wie weit Mira
vorgedrungen ist, und wie es um die Architekten der Stadt und im Amt gestellt
ist.

Vor diesem Hintergrund beschliessen Alessia und Tim gemeinsam Widerstand zu
leisten und eine Bewegung ins Leben zu rufen, die die K.I. eindämmen
soll. Natürlich soll Jan auch mitmachen, denn er hat Zugang zu andern Kreisen
junger Leute, wo Alessia und Tim weniger zugriff haben. Als sie Jan ihre
Absichten erklären zeigt dieser den beiden schuldbewusst, dass er die ganze
Zeit über Mira auf seinem Computer am laufen hatte.

\section*{Episode 6 | Tim}

Die sechste Episode wird aus der Perspektive von \emph{Tim Bergmann} erzählt.

Nachdem sich Jan am Ende der letzten Episode den Tim und Alessia anvertraut
hatte, war in ihrem Kreis der ehemaligen besten Freunde eine eisige Kälte
eingezogen. Alessia hatte Jan indirekt für alles verantwortlich gemacht, was
Passiert war. Tim, dem an der Freundschaft mit beiden viel liegt hat sich in
die Rolle des Vermittlers begeben, um möglichst viel Glut aus dem Feuer zu
ziehen, so lange dies noch geht, und ihre Freundschaft noch keinen ireparaben
Schaden genommen hatte. Auch wenn es in Zukunft vermutlich nie mehr ganz so
sein würde, wie es vorher gewesen war. Die unbeschwerte Lockerheit würde wohl
nie wieder in diesem Masse zurückkehren.

Als Tim Alessia endlich davon überzeugt, dass ihr Schmollen nichts bringen
wird für ihre Zukunft gelingt es ihm die kleine Gruppe wieder zu
vereinen. Jan hat ein schlechtes Gewissen, da er sich auch mitverantwortlich
fühlt für alles was passiert ist, und möchte darum alles in seiner Macht
stehende tun, um eine Gegenbewegung zu lancieren. Die drei versuchen nun also
nach Anlaufschwierigkeiten sich zu sammeln und zu überlegen, was man denn
konkret tun kann, um die Situation zu verändern. Sie kommen gemeinsam zu dem
Schluss, dass mit Marginalitäten hier nichts auszurichten sei, und
beschliessen daher, dass sie Anschläge auf Mira ausüben wollen um sie
letztendlich auszuschalten. Dies meinen die drei auf die wörtlichste Weise.

Tim der Hochschulpolitisch aktiv ist hat einen guten Zugang zu den Studenten,
und vermag es mit seiner Eloquenz und seinen guten Argumenten aus der bei
allen Studenten gedrückten Stimmung Kapital zu schlagen und die meisten von
ihnen hinter die Bewegung zu sammeln. Sie diskutieren in einer grossen Gruppe
abends im Hörsaal, wie denn die Anschläge auf etwas nicht physisches aussehen
könnten. Leider fehlt allen ein tieferes Verständnis dafür, wie eine
Künstliche Intelligenz wirklich funktioniert, um eine richtige Schwachstelle
zu finden. Nichtsdestotrotz sind alle guten Mutes, dass sie gemeinsam etwas
bewirken können.

Neben den ``physischen'' Anschlägen wollen die Studenten gemeinsam mit ihrer
Bewegung politischen Druck ausüben, und so eine nachhaltigere Lösung
schaffen, die es künstlichen Intelligenzen verbieten soll, mehr zu machen als
die richtigen Parameter zu finden und in Einklang zu bringen. Alle
Entscheidungsfreiheit soll künftig wegfallen.

Aber letztenendes Fruchten weder die Anschläge auf Mira, noch finden sie
sonderlich offene Ohren in der Politik, da die meisten Menschen davon
überzeugt sind, dass die K.I. der richtige Weg sei. Es konnten bisher
Unsummen an Geld eingespart und anderweitig ausgegeben werden.

Mit dem Fehlschlag der Bewegung macht sich nun allgemein eine Resignation bei
den jungen Architekten breit. Aber Tim vermag es noch einmal alle zu
motivieren und vom weitermachen zu überzeugen.

Gemeinsam halten die Studenten unter Tims Feder noch einmal eine lange
Krisensitzung ab, die so lange dauert, dass die Studenten die ganze Nacht
gemeinsam am Hönggerberg verbringen.

Am nächsten Morgen wird bekannt, dass sich ihr Problem möglicherweise von
selber lösen wird. In ihrem rationalisierenden und effizienten Ansatz, möchte
Mira so viel wie möglich in so kurzer Zeit als möglich realisieren. Dies
führt letztendlich dazu, dass Zürich nur noch eine einzige Baustelle ist, und
die Infrastruktur zum erliegen kommt.

Die Episode Schliesst mit dem Bild, wo man Zürich als Baustelle aus der
Vogelperspektive sieht und erkennt, dass sonst nichts mehr geht.

\section*{Episode 7 | Professor Brändi}

Die siebte Episode wird aus der Perspektive von \emph{Professor Arno Brändi}
erzählt.

Professor Brändi steht wie gewohnt morgens auf, trinkt seinen Kaffee und
macht sich auf den Weg richtung ETH. In der gesamten Stadt aber ist der
Verkehr zum erliegen gekommen. Das einzige was noch funktioniert ist der
Fernverkehr mit der Bahn. Dies hilft Brändi aber wenig, da er durch die Stadt
muss um zum Hönggerberg zu gelangen. In seiner aufgestellten, sanguinischen
Art verzagt er nicht, und geht zu Fuss los.

Nur eine Stunde zu späht kommt Brändi an der ETH an, und ist überrascht, dass
seine Studenten schon alle vollzählig erschienen sind. Er erzählt von seinem
Erlebnis in der Stadt, und ist erstaunt, wie es alle Studenten scheinbar
pünktlich zum Unterricht geschafft haben. Dies erfüllt ihn aber ehrlich mit
Freude. Brändi arbeitet äusserst gerne mit interessierten jungen Leuten
zusammen.

Zu seiner Verwunderung aber wollen die Studenten heute keinen gewöhnlichen
Unterricht abhalten, sondern möchten sich mit Brändi über die aktuellen
Geschehnisse beraten.

Mit einer Ellipse sieht man, wie sich in den fünf folgenden Tagen eigentlich
nichts geändert hat. Die Stadt liegt immer noch lahm da. Die Menschen haben
jedoch begonnen sich anzupassen. Mittlerweile sind viele Brändis Beispiel
gefolgt und bewegen sich zu Fuss oder auf dem Fahrrad durch die Stadt. Die
Strassen die vorher vollgepackt mit Autos waren sind nun eine grosse
Fussgängerzone geworden.

Brändi hat mit den Studenten ausgemacht, dass sie gemeinsam versuchen werden
etwas auszurichten, obwohl es Brändi nicht sonderlich stört, die Stadt von
den Autos befreit zu sehen. Sie werden gemeinsam versuchen Dr. Brown
ausfindig zu machen, den man seit dem offenkundigen Scheitern Miras nicht
mehr gesehen hatte. Zudem ist der Weg, das Bauvorhaben von Mira mittels
Mangel an Zulieferung zu stoppen, oder mindestens einzudämmen, ein
vielversprechender, den sich die Studenten gar nicht überlegt gehabt
hatten. So wollen sie die ohnehin schon prekäre Situation der Versorgung der
Baustellen noch künstlich verknappen.

In zwei Detachementen versuchen die Studenten also wirksam zu werden. Nach
langem Suchen und recherchieren finden die Studenten, die mit Brändi
unterwegs durch die ganze Stadt ziehen Dr. Brown. Brown wollte erst wieder
auftauchen, wenn er eine Verbesserung für Mira bereit hat, die eine Solche
Situation unmöglich macht.

Nach intensivem Einreden von Brändi auf Brown willigt dieser endlich ein, den
Studenten zu helfen, und für sie enen Patch für Mira zu schreiben, der Mira
einschränken soll. Nach nur einem Tag kommt er mit dem fertigen Patch zu
Brändi und gibt diesem Anweisungen, wie man das update einspielen kann. Durch
das Upgrade soll Mira letztendlich keine alleinige Entscheidungsgewalt mehr
haben.

Als die Studenten nun mira endlich eingedämmt haben, möchten sie das Projekt
sofort zerstören, doch Brändi gibt den Input, dass statt Mira zu zerstören,
sie einen Weg finden müssen, sich mit ihr zu arrangieren. Es werden
schliesslich auch neue künstliche Intelligenzen geschaffen werden, wo sie
keinen Einfluss darauf haben werden. Er appeliert daran, dass sich die
Studenten darauf besinnen, was ihre Vorzüge gegenüber einer Maschine sind,
wie sie also niemals überflüssig gemacht werden können, und gibt ihnen auch
den Anstoss sich zu überlegen, wie sie von einer K.I. profitieren können. Nur
so könne eine nachhaltig gedachte Zukunft aussehen, wenn man sich nicht gegen
sie auflehnt, sondern sie mitgestaltet.

\section*{Episode 8 | Gudzilla}

die achte und letzte Episode der ersten Saison wird aus der Perspektive von
\emph{ETH Präsident Lino Gudzilla} erzählt.

Nachdem in den Medien das gewaltige Ausmass des Scheiterns vom Projekt Mira
diskutiert wird und somit auch die Reputation der ETH angegriffen ist,
entschliesst sich Gudzilla Dr. Brown zu entlassen, und dies öffentlich zu
demonstrieren. Man wolle nicht, dass sich kriminelle Elemente aim Lehrkörper
der ETH befinden. So wird Brown offiziell angeprangert, Mira gestohlen zu
haben, was ja die internen Untersuchungen der ETH ergeben hatten.

Da in den Forschungslaboratorien geheimhaltung herrscht, konnte die Polizei
bei ihren Ermittlungen aus Mangel an Informationen nicht zum gleichen Schluss
kommen. So wird aber Dr. Browns Entlassung öffentlich auch als fadenscheinig
angeprangert und lastet schwer auf den Schultern des amtierenden
ETH-Präsidenten. Zwar argumentiert er wahrheitsgemäss, doch kann er
öffentlich keine Argumente vorlegen.

Um der laufenden Abwärtsspirale Herr zu werden, ernennt Gudzilla den
beliebtesten Mann des Lehrkörpers, Prof. Arno Brändi, zum Dekan der Fakultät
der Architektur, um die ja das ganze Aufsehen ist, und beauftragt ihn mit der
Umstrukturierung der Lehre und des Departementes an sich, um einen
zukunftsweisenden Weg zu finden.

Unter der Federführung von Brändi erholt sich die Reputation der ETH
erstaunlich schnell. Man lobt den Umgang mit den neuen Möglichkeiten und dem
festhalten am bestehenden. Brändi scheint das Problem so gut anzugehen, dass
Gudzilla so quasi aus dem Schneider kommt.

Als nun Gudzilla der festen Überzeugung ist, dass sich die Wogen nun
endgültig geglättet haben, stirbt Brändi plötzlich bei einem tragischen
Unfall. Da nun der Mann der Stunde tot ist, muss Gudzilla schleunigst wieder
selber aktiv werden.

Dummerweise findet er niemanden, der die entstandene Lücke auch nur
ansatzweise so gut füllen könnte, wie dies Brändi getan hatte. Er möchte aber
nicht neue Unzufriedenheit streuen und vorschnell jemanden einsetzen, der am
Schluss mehr schaden anrichten könnte als bisher schon geschehen war.

Parallel dazu bekommt Giovanni Benini vom Präsidium der Stadt Zürich eine
neue Arbeitsstelle angeboten. Er soll künftig das Amt für den Städtebau als
Direktor anführen. Giovanni ist aber nicht im mindesten an der neuen Stelle
interessiert. Er hat nicht einfach vergessen, wie er vor kurzer Zeit einfach
abserviert wurde, und möchte nichts mehr mit seinem alten Arbeitgeber zu tun
haben.

Als Gudzilla eine Berichterstattung darüber sieht, ist er sich sicher, den
richtigen Mann für die Stelle gefunden zu haben. Er beruft Giovanni zum
Professor und setzt diesen gleich in das Amt des Dekans ein.

Im Rahmen der Antrittsvorlesung für Giovanni lässt Gudzilla nochmals alle
Ereignisse der vergangenen Zeit revue passieren. Im folgenden scheint ein
vollends harmonischer Umgang mit der K.I. gefunden worden zu sein, wo deren
Potenzial genutzt wird, sie sich aber nicht über die Menschen hinweg setzten
kann.

Als letztes Bild sieht man, wie Dr. Brown in einem teuren Luxusauto im
sonnigen Kalifornien herumfährt und einen Anruf entgegen nimmt. Der Mann am
Apparat, offenbar persönlicher Sekretär des CEO fragt nach, was er denn für
die Präsentation von Mira 2.0 benötige\ldots{}
\end{document}
